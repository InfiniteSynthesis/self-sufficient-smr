\subsection{Full Protocol Description}
\label{subsec:full-protocol-description}

\paragraph{Main protocol instance.}
%
We introduce the main \pSMR protocol instance that dispatches to the relevant subprocesses following the UC notions.

\begin{cccProtocol}
    {$\pSMR(\party, \sid; \funcDriftingClock, \funcRO, \funcDiffuse, \funcCRS)$}
    {main-protocol}
    {The main protocol instance of \pSMR.}

    \paragraph{Global Variables:}
    %
    \begin{cccItemize}[nosep]
        \item Read-only: \syncLen, \epochLen, $t_{\mathsf{off}}$, $t_{\mathsf{gather}}$

        \item Read-write: \localTime, \epoch, \parallelChainsLocal, \parallelTreesLocal, $T^{\epoch}$, $\mathtt{isInit}$, $t_{\mathsf{work}}$, \buffer, \inputBlockBuffer, \futureChains, \isSync, \fetchCompleted, $\mathsf{arrivalTime}(\cdot)$, \state, $\mathtt{snapshot}$
    \end{cccItemize}

    \paragraph{Registration / Deregistration:}
    %
    \begin{cccItemize}[nosep]
        \item Upon receiving input $(\textsc{register}, \mathcal{R})$, where $\mathcal{R} \in \{\funcLedger, \funcDriftingClock \}$ execute protocol $\mathsf{Registration}(\party,\allowbreak \sid, \mathtt{reg}, \mathcal{R})$.

        \item  Upon receiving input $(\textsc{de-register}, \mathcal{R})$, where $\mathcal{R} \in \{\funcLedger, \funcDriftingClock \}$ execute protocol $\mathsf{Deregistration}(\party, \sid,\allowbreak \mathtt{reg}, \mathcal{R})$.

        \item Upon receiving input $(\textsc{is-registered}, \sid)$ return $(\textsc{register}, \sid, 1)$ if the local registry $\mathtt{Reg}$ indicates that this party has successfully completed a registration with $\mathcal{R} = \funcLedger$ (and did not de-register since then).
        %
        Otherwise, return $(\textsc{register}, \sid, 0)$.
    \end{cccItemize}

    \paragraph{Interacting with the Ledger:}
    %
    Upon receiving a ledger-specific input $I \in \{(\textsc{submit}, \ldots), \allowbreak (\textsc{read}, \ldots), \allowbreak (\textsc{maintain-ledger}, \ldots) \}$ verify first that all resources are available.
    %
    If not all resources are available, then ignore the input; else (i.e., the party is operational and time-aware) execute one of the following steps depending on the input $I$:
    %
    \begin{cccItemize}[nosep]
        \item \textbf{If} $I = (\textsc{submit}, \sid, \tx)$ \textbf{then} set $\buffer \gets \buffer \concat \tx$, and send $(\textsc{diffuse}, \sid, \tx)$ to $\funcDiffuse^{\textsf{tx}}$.

        \item \textbf{If} $I = (\textsc{maintain-ledger}, \sid, \mathrm{minerID})$ \textbf{then} invoke protocol $\mathsf{LedgerMaintenance}(\party, \sid)$; \textbf{if} $\mathsf{LedgerMaintenance}$ halts \textbf{then} halt the protocol execution (all future input is ignored).

        \item \textbf{If} $I = (\textsc{read}, \sid)$ then invoke protocol $\mathsf{ReadState}(\party, \sid)$.

        \item \textbf{If} $I = (\textsc{export-time}, \sid)$ then do the following: if \isSync or $\mathtt{isInit}$ is false, then return $(\textsc{export-time}, \sid, \bot)$; otherwise call $\mathsf{UpdateLocalTime}(\party, \sid)$ and return $(\textsc{export-time}, \sid, \localTime)$ to the caller.
    \end{cccItemize}

    \paragraph{Handling calls to the shared setup:}
    %
    \begin{cccItemize}[nosep]
        \item Upon receiving $(\textsc{clock-tick}, \sid_C)$, forward it to \funcDriftingClock and output \funcDriftingClock's response.

        \item Upon receiving $(\textsc{clock-update}, \sid_C)$, record that a \textsc{clock-update} was received in the current round.
        %
        If the party is registered to all its setups, then do nothing further.
        %
        Otherwise, do the following operations before concluding this round:
        %
        \begin{enumerate}[label=\arabic*,leftmargin=*, nosep]
            \item If this instance is currently time-aware but otherwise stalled or offline, then call $\mathsf{UpdateLocalTime}(\party, \sid)$ to update \localTime.
                  %
                  If the party has passed a synchronization slot, then set $\isSync \gets \false$.

            \item If this instance is only stalled but $\isSync = \true$, then additionally execute $\mathsf{FetchInformation}(\party, \sid)$, extract all new input blocks (synchronization beacons) $IB$ from the fetched chains and record their arrival times and set $\fetchCompleted \gets \true$.
                  %
                  Also, any unfinished interruptible execution of this round is marked as completed.

            \item Forward $(\textsc{clock-update}, \sid_C)$ to \funcDriftingClock to finally conclude the round.
        \end{enumerate}
    \end{cccItemize}
\end{cccProtocol}

\paragraph{Registration / de-registration.}
%
In order to perform basic operations, a party \party needs to register to all resources.
%
Note that the protocol will initialize local time $\party.\localTime$ to \protocolTime{1}{1}, and \party is aware whether he is not synchronized not and will set the bit variable \isSync correspondingly.

\begin{cccProtocol}
    {$\mathsf{Registration}(\party, \sid, \mathtt{Reg}, \mathcal{G})$}
    {registration}
    {Parties register on necessary resources (functionalities) to run the protocol.}

    \begin{algorithmic}[1]
        \oneLineIf{$\mathcal{G} = \funcDriftingClock$}{send $(\textsc{register}, \sid)$ to $\mathcal{G}$, set registration status to registered with $\mathcal{G}$, and output the valued received by $\mathcal{G}$.}

        \If{$\mathcal{G} = \funcLedger$}

        \If{\party is not registered with \funcDriftingClock or already registered with all setups}

        \State{ignore this input}

        \Else

        \State{Send $(\textsc{clock-tick}, \sid_C)$ to \funcDriftingClock and receive $(\textsc{clock-tick}, \sid_C , \mathrm{tick})$}

        \State{Send $(\textsc{register}, \sid)$ to \funcDiffuse}

        \State{$\localTime \gets \protocolTime{1}{1}$ and $\isSync \gets \false$}

        \State{If this is the first registration invocation for this ITI, then set $\mathtt{isInit} \gets \false$.}

        \State{Output $(\textsc{register}, \sid, \party)$ once completing the registration with all the above resources \func}
        \EndIf
        \EndIf
    \end{algorithmic}
\end{cccProtocol}

The deregistration process is an analogous action that sets variables to the initial values.

\begin{cccProtocol}
    {$\mathsf{Deregistration}(\party, \sid, \mathtt{Reg}, \mathcal{G})$}
    {deregistration}
    {Parties de-register from corresponding resources.}

    \begin{algorithmic}[1]
        \If{$\mathcal{G} = \funcDriftingClock$}

        \State{Set $\isSync \gets \false$}

        \State{Send $(\textsc{de-register}, \sid)$ to $\mathcal{G}$ and set registration status as de-registered with $\mathcal{G}$}

        \State{Output the valued received by $\mathcal{G}$}

        \EndIf

        \If{$\mathcal{G} = \funcLedger$}

        \State{Set $\isSync \gets \false$}

        \State{Send $(\textsc{de-register}, \sid)$ to \funcDiffuse, set its registration status as de-registered with \funcDiffuse and output $(\textsc{de-register}, \sid, \party)$.}
        \EndIf
    \end{algorithmic}
\end{cccProtocol}

\paragraph{Ledger maintenance.}
%
The protocol $\mathsf{LedgerMaintenance}$ groups all the steps regarding the main ledger operation.
%
Note that, depending on a party is alert or not, she might execute different sub protocols.
%
For parties that are not synchronized, after querying \funcCRS, they first enter the bootstrapping mode by calling $\mathsf{JoiningProcedure}$.
%
By executing this sub protocol, they set their internal state \isSync to true and then start to execute the normal ledger maintenance operations.

\begin{cccProtocol}
    {$\mathsf{LedgerMaintenance}(\party, \sid)$}
    {ledger-maintenance}
    {The main operations for parties to maintain the ledger.}

    \begin{algorithmic}[1]
        \LineComment{The following steps are executed in an $(\textsc{maintain-ledger}, \sid, \mathrm{minerID})$-interruptible manner:}

        \If{$\mathtt{isInit} = \false$}
        \State{Send $(\textsc{Retrieve, \sid})$ to \funcCRS and set response as \textsf{CRS} and set $\mathtt{isInit} \gets \true$.}
        \EndIf

        \LineComment{Bootstrap if not synchronized.}
        \oneLineIf{\textbf{not} \isSync}{Call $\mathsf{JoiningProcedure}(\party, \sid)$}
        \Comment{\cref{protocol:joining-procedure}}

        \LineComment{Normal operations when alert.}

        \State{Invoke $\mathsf{FetchInformation}(\party, \sid)$ and denote the output by $(\chain_1, \ldots, \chain_N)$, $(\tx_1, \ldots, \tx_k)$}
        \State{$\buffer \gets \buffer \concat (\tx_1, \ldots, \tx_k)$ and $\futureChains \gets \futureChains \cup \{ \chain_1, \ldots, \chain_N \}$}

        \State{Call $\mathsf{UpdateLocalTime}(\party, \sid)$}
        \Comment{\cref{protocol:update-local-time}}

        \LineComment{Ensures the processing of new input-blocks arrived in chains only.}
        \State{Extract input-blocks $IB \gets \{ \inputBlock_1, \ldots , \inputBlock_n \}$ contained in $\chain_1, \ldots, \chain_N$ and not yet contained in \inputBlockBuffer.}

        \State{Call $\mathsf{ProcessInputBlocks}(P, \sid, IB)$}
        \Comment{\cref{protocol:process-input-blocks}}

        \State{Let $\mathcal{N}_0$ be a set of (single) chains s.t. $\chain \in \mathcal{N}_0 :\Leftrightarrow \chain \in \parallelChains \in \futureChains \wedge \forall \block \in \chain : \timestamp{\block} \le \localTime$}

        \State{Remove each $\chain \in \mathcal{N}_0$ from \futureChains}

        \State{$\fetchCompleted \gets \true$}

        \State{Call $\mathsf{UpdateLocalChain}(\parallelChainsLocal, \parallelTreesLocal, \mathcal{N}_0)$ to update \parallelChainsLocal and \parallelTreesLocal}

        \If{$t_{\mathsf{work}} < \localTime$}
        \State{Call $\mathsf{MiningProcedure}(\party, \sid, r)$}
        \Comment{\cref{protocol:mining-procedure}}

        \If{$\round = \interval \cdot \syncLen$}
        \State{Call $\mathsf{StateUpdate}(\party, \sid)$}
        \Comment{\cref{protocol:update-state}}
        \State{Call $\mathsf{SyncProcedure}(\party, \sid)$}
        \Comment{\cref{protocol:sync-procedure}}
        \EndIf

        \State{Set $t_{\mathsf{work}} \gets \localTime$}

        \EndIf

        \State{Call $\mathsf{FinishRound}(\party)$}
        \Comment{Mark normal round actions as finished}
    \end{algorithmic}
\end{cccProtocol}


\paragraph{Fetch information.}
%
Parties fetch block information from $\funcDiffuse^{\textsf{bc}}$ to learn new parallel chains (precisely, blocktrees) with possibly future timestamps.
%
Note that in order to simplify the chain validation and selection procedure, we let $\mathsf{FetchInformation}$ return all single chains $\chain_1, \ldots, \chain_N$ extracted from all received blocktrees\footnote{Parties associate all blocktrees/chains with their index $i \in [m]$ in parallel chains(trees), and thus apply chain validation and validation for chains with the same index. For brevity, we ignore these details in our protocol description and assume all chains are processed correspondingly.}.
%
Parties also fetch transactions from $\funcDiffuse^{\textsf{tx}}$ to learn new transactions.

\begin{cccProtocol}
    {$\mathsf{FetchInformation}(\party, \sid)$}
    {fetch-information}
    {Fetching new blocks and transactions from the diffusion functionality.}

    \begin{algorithmic}[1]
        \oneLineIf{\fetchCompleted}{\Return}
        \Comment{Fetch once per round and never catch up missed round.}

        \LineComment{Fetch blocks on $\funcDiffuse^{\textsf{bc}}$}

        \State{Send $(\textsc{fetch}, \sid)$ to $\funcDiffuse^{\textsf{bc}}$; denote the response by $(\textsc{fetch}, \sid, bc)$.}

        \State{Extract blocktrees $\blockTree_1, \ldots \blockTree_n$ from $bc$}

        \State{Extract chains $\chain_1, \ldots \chain_N$ from $\blockTree_1, \ldots \blockTree_n$}

        \LineComment{Fetch transactions on $\funcDiffuse^{\textsf{tx}}$}

        \State{Send $(\textsc{fetch}, \sid)$ to $\funcDiffuse^{\textsf{tx}}$; denote the response by $(\textsc{fetch}, \sid, tx)$.}

        \State{Extract transactions $(\tx_1, \ldots \tx_k)$ from $tx$}

        \If{\textbf{not} \isSync or \party is stalled}
        \State{$\buffer \gets \buffer \concat (\tx_1, \ldots \tx_k)$}
        \State{$\futureChains \gets \futureChains \cup \{ \parallelChains_1, \ldots \parallelChains_N \}$}
        \EndIf
    \end{algorithmic}

    \textsc{Output:} The protocol outputs $(\chain_1, \ldots , \chain_N)$ and $(\tx_1, \ldots, \tx_k)$ to its caller (but not to \Z).
\end{cccProtocol}

\paragraph{Chain and input block validation.}
%
We present the chain and input block validation procedure in \pSMR respectively.
%
Note that, different from the single chain protocols where the validity of each chain can be attested independently, chains in \pSMR need to be verified per interval (see the chain selection procedure in~\cref{algorithm:update-local-chain}).
%
Hence the algorithm $\mathsf{IsValidChain}$ takes as input a chain \chain, an integer $i$ the index of \chain in parallel chains, a parallel blocktree \parallelTrees that \chain is associate with and an integer $itvl$ the target interval on \chain to be verified.

The blocks (input-blocks) in \pSMR are of form
%
\[ \block = \langle ctr, \protocolTime{itvl}{r}, h, st, h^\ast, h', val \rangle, \]
%
where $ctr$ is the nonce for PoW, \protocolTime{itvl}{r} shows the block timestamp, $h$ is the hash pointer to the previous block, $st$ represents the block content (note that $h$ and $st$ are meaningful only for a block \block), $h^\ast$ is the fresh randomness when mining input-blocks, and $h', val$ is the chain reference for previous intervals (cf.~\cref{subsec:new-parallel-blockchain}) and input-block content respectively.

The following functions help us simplify the validation process:
%
\begin{cccItemize}[noitemsep]
    \item We use \textsf{ValidBlock} to verify if a block is a successful PoW on the $i$-th chain (that is, the nonce $ctr$ is valid and the block hash --- $i$-th segment of the RO output is less than target $T$).
    %
    \[ \mathsf{ValidBlock}(\block, i, T) \triangleq \stringSegment{H(\block)}{i}{m} < T \wedge \block.ctr < 2^{32}. \]

    \item Analogously, we use use \textsf{ValidInputBlock} to verify if a synchronization beacon is a successful PoW on its associated chain by checking the reverse of the string segment.
    %
    I.e.,
    %
    \begin{equation*}
        \mathsf{ValidInputBlock}(\inputBlock, i, T) = \stringSegmentRev{H(\inputBlock)}{i}{m} < T \wedge \inputBlock.ctr < 2^{32}
    \end{equation*}

    \item We adopt \textsf{isStageOG} as a predicate to justify if a timestamp $r$ is in the output generation stage of interval $itvl$.
    %
    Precisely,
    %
    \begin{equation*}
        \mathsf{isStageOG}(itvl, r) \triangleq r \in [(itvl - 1) \cdot \syncLen + \syncLen_{\mathsf{VC}}, itvl \cdot \syncLen - \syncLen_{\mathsf{RC}}].
    \end{equation*}
\end{cccItemize}

Given the above, \cref{algorithm:isvalidchain} presents a full specification of the chain validation procedure.

\begin{cccAlgorithm}
    {$\mathsf{IsValidChain}(\chain, i, \parallelTrees, itvl)$}
    {isvalidchain}
    {The chain validation procedure.}

    \newcommand*{\goodHash}{\mathEnv{\mathsf{goodHash}}}
    \newcommand*{\goodNonce}{\mathEnv{\mathsf{goodNonce}}}
    \newcommand*{\goodTime}{\mathEnv{\mathsf{goodTime}}}
    \newcommand*{\goodInputBlock}{\mathEnv{\mathsf{goodInputBlock}}}
    \newcommand*{\goodCRS}{\mathEnv{\mathsf{goodCRS}}}

    \begin{algorithmic}[1]
        \LineComment{This algorithm has five internal Boolean variables \goodHash, \goodNonce, \goodTime, \goodCRS and \goodInputBlock, all initialized as \true.}

        \oneLineIf{\chain starts with a block with hash reference other than $\mathsf{CRS}$}{$\goodCRS \gets \false$}

        \oneLineIf{\isSync \textbf{and} $\exists \block \in \chain$ s.t. $\mathsf{validOrder}(\timestamp{\block}, \localTime) = \false$}{$\goodTime \gets \false$}

        \LineComment{Derive fresh randomness for $itvl$ as indicated by \chain}
        \State{Extract $h^\ast_{itvl}$ from $\chainHead{\chainPrefixUB{\chain}{(itvl - 1) \cdot \syncLen + \syncLen_{\mathsf{VC}}}}$}

        \For{each block \block in \chain from interval $itvl$}

        \State{Parse \block as $\langle ctr, \protocolTime{itvl}{r}, h, st, \cdot, \cdot, \cdot \rangle$}

        \State{Set $T_\block \gets \mathsf{ExtractMiningTarget}(\chain, \lceil itvl / (\epochLen / \syncLen) \rceil)$.}
            \Comment{Extract mining difficulty set for \block}

            \LineComment{Check hash}
            \State{Let $\block^{-1}$ be the last block in \chain before \block}
            \State{Set $\goodHash \gets (\stringSegment{h}{i}{m} = \stringSegment{H(\block^{-1})}{i}{m})$}

            \LineComment{Check nonce}
            \State{Set $\goodNonce \gets \mathsf{ValidBlock}(\block, i, T_\block)$}

            \LineComment{Check input blocks}

            \If{$\neg \mathsf{isStageOG}(itvl, r)$ \textbf{and} $\exists \inputBlock \in \block$}
            \State{Set $\goodInputBlock \gets \false$}
            \Comment{Block should include \inputBlock only in \textsf{OG}}
            \Else

            \For{each input block $\inputBlock \in \block$}
            \State{Parse \inputBlock as $\langle ctr', \protocolTime{itvl'}{r'}, \cdot, \cdot, h^\ast_\inputBlock, \cdot, \cdot \rangle$}

            \oneLineIf{$\mathsf{isStageOG}(itvl, r) = \false$ \textbf{or} $\timestamp{\inputBlock} > \timestamp{\block}$}{$\goodInputBlock \gets \false$}

            \oneLineIf{\chain contains duplicate \inputBlock \textbf{or} $\stringSegment{h^\ast_\inputBlock}{i}{m} \neq \stringSegment{h^\ast_{itvl}}{i}{m}$}{$\goodInputBlock \gets \false$}

            \State{Set $T_\inputBlock \gets \mathsf{ExtractMiningTarget}(\chain, \lceil itvl' / (\epochLen / \syncLen) \rceil)$.}
        \State{Set $\goodInputBlock \gets \mathsf{ValidInputBlock}(\inputBlock, i, T_{\inputBlock})$}

        \EndFor

        \EndIf

        \EndFor

        \oneLineIf{$\goodHash \wedge \goodNonce \wedge \goodInputBlock \wedge \goodTime \wedge \goodCRS$}{\Return \true}
        \State{\Return \false}
    \end{algorithmic}
\end{cccAlgorithm}

Regarding input blocks that have not yet been included in chains, the validity of an \inputBlock with respect to a chain index $i$ is checked only when the party's local time has forwarded to at least the same interval as \inputBlock (if not, this procedure returns ``undecided'' which defers the validity check).

\begin{cccAlgorithm}
    {$\mathsf{IsValidInputBlock}(\inputBlock, \chain, i, \parallelTrees)$}
    {isValidInputBlock}
    {The input-block validation procedure.}

    \begin{algorithmic}[1]
        \LineComment{Precondition: Chain \chain is valid. Returns \true if the beacon is a valid beacon w.r.t. \chain, $\mathsf{undecided}$ if no judgement is possible, and \false if the beacon is invalid w.r.t. \chain.}

        \State{Parse \inputBlock as $\langle ctr, \protocolTime{itvl}{r}, \cdot, \cdot, h^\ast_\inputBlock, \cdot, \cdot \rangle$}

        \oneLineIf{\chain contains no block in interval $itvl$}{\Return $\mathsf{undecided}$}
        \Comment{no judgement possible}

        \LineComment{Check nonce value and freshness}
        \State{Extract $h^\ast_{itvl}$ from $\chainHead{\chainPrefixUB{\chain}{(itvl - 1) \cdot \syncLen + \syncLen_{\mathsf{VC}}}}$}
        \State{Set $T_\inputBlock \gets \mathsf{ExtractMiningTarget}(\chain, \lceil itvl / (\epochLen / \syncLen) \rceil)$.}
        \oneLineIf{$\mathsf{ValidInputBlock}(\inputBlock, i, T_{\inputBlock}) = \true$ \textbf{and} $\stringSegment{h^\ast_\inputBlock}{i}{m} \neq \stringSegment{h^\ast_{itvl}}{i}{m}$}{\Return \true}

        \State{\Return \false}
    \end{algorithmic}
\end{cccAlgorithm}

\paragraph{Extract mining target.}
%
We adopt algorithm $\mathsf{ExtractMiningTarget}$ to calculate the mining difficulty set for an epoch index $ep$ on a single chain \chain.
%
Starting from the initial target $T_0$, this algorithm iteratively calculates targets from the first epoch to the targeted one, based on the number of blocks in each epoch and \cref{eq:target-recalc}.

\begin{cccAlgorithm}
    {$\mathsf{ExtractMiningTarget}(\chain, ep)$}
    {extract-mining-target}
    {Extracting the mining target for a given epoch.}

    \begin{algorithmic}[1]
        \oneLineIf{$ep = 1$}{\Return $T_0$} \Comment{Return initial target}
        \For{$i$ \textbf{from} $2$ to $ep$}
        \State{$\varLambda_i = |\{ \block \mathbin| \block \in \chain \wedge \lceil \timestamp{\block} / (\epochLen / \syncLen) \rceil = i \}|$}

        \State{$T_i = \min \{ \max \{ \varLambda_{\mathsf{epoch}} / \varLambda_i \cdot T_{i - 1}, 1 / \tau \}, \tau \}$}
        \Comment{\cref{eq:target-recalc}}
        \EndFor
        \State{\Return $T_{ep}$}
    \end{algorithmic}
\end{cccAlgorithm}

\paragraph{Update local time.}
%
Parties will send \textsc{clock-tick} to \funcDriftingClock to check if it receives a $\mathrm{tick} = 0$, which indicates the beginning of a new (local) round.

\begin{cccProtocol}
    {$\mathsf{UpdateLocalTime}(\party, \sid)$}
    {update-local-time}
    {Parties update their local clock due to \funcDriftingClock.}
    
    \begin{algorithmic}[1]
        \LineComment{Precondition: Only executed if time-aware.}
        \State{Send $(\textsc{clock-tick}, \sid)$ to \funcDriftingClock and receive $(\textsc{clock-tick}, \sid , \mathrm{tick})$}

        \If{$\mathrm{tick} = 0$}
        \State{$\localTime \gets \localTime + 1$}
        \State{$\fetchCompleted \gets \false$}
        \EndIf
        \State{$\epoch \gets \lceil \interval / (\epochLen / \syncLen) \rceil$}
    \end{algorithmic}

    \textsc{Output:} The protocol outputs \localTime, \epoch to its caller (but not to \Z)
\end{cccProtocol}

\paragraph{Process input blocks and arrival times.}
%
The following procedure processes incoming input blocks, bookkeeps their arrival times and filters out duplicate ones.
%
The predicate to verify input-block validity is presented in~\cref{algorithm:isValidInputBlock}.
%
Regarding the duplicate beacons, only one with the earliest arrival time will be preserved.

\begin{cccProtocol}
    {$\mathsf{ProcessInputBlocks}(\party, \sid, IB)$}
    {process-input-blocks}
    {Parties filter invalid input blocks.}

    \begin{algorithmic}[1]
        \oneLineIf{$\fetchCompleted = \true$}{\Return}

        \State{Send $(\textsc{fetch}, \sid)$ to $\funcDiffuse^{\textsf{input}}$ and denote the response by $(\textsc{fetch}, \sid, b)$}

        \State{Extract all received input blocks $(\inputBlock_1, \ldots , \inputBlock_k)$ contained in $b \cup IB$.}

        \For{each $\inputBlock_i$ with $\mathsf{arrivalTime}(\inputBlock) = \bot$}
        \State{$\inputBlockBuffer \gets \inputBlockBuffer \cup \{ \inputBlock \}$}

        \State{Parse $\timestamp{\inputBlock}$ as \protocolTime{itvl'}{r'}}

        \If{$\isSync \wedge (\interval \ge itvl')$}

        \State{Set $\mathsf{arrivalTime}(\inputBlock_i) \gets (\localTime, \mathsf{final})$}
        \Comment{The measurement is final.}

        \Else \Comment{Will be adjusted upon next time shift.}
        \State{$\mathsf{arrivalTime}(\inputBlock_i) \gets (\localTime, \mathsf{temp})$}

        \EndIf
        \EndFor

        \LineComment{Buffer cleaning.}

        \If{\isSync}

        \For{each $\inputBlock \in \inputBlockBuffer$}

        \oneLineIf{\timestamp{\inputBlock} has timestamp later than \interval}{skip \inputBlock}

        \State{$\mathsf{goodInputBlock} \gets \false$}

        \For{$i$ \textbf{from} $1$ \textbf{to} $m$}

        \oneLineIf{$\exists \chain \in \parallelTreesLocal$ s.t. $\mathsf{IsValidInputBlock}(\inputBlock, \chain, i, \parallelTreesLocal) = \true$}{$\mathsf{goodInputBlock} \gets \true$}

        \EndFor

        \oneLineIf{$\mathsf{goodInputBlock} = \false$}{Remove \inputBlock from \inputBlockBuffer}

        \EndFor

        \EndIf
    \end{algorithmic}

    \textsc{Output:} The protocol outputs \ok to its caller (but not to \Z).
\end{cccProtocol}

\paragraph{Chain selection.}
%
Our chain selection rule (which should only be used by time-aware parties) works in two steps.
%
First, it filters all invalid chains, by verifying them in an interval-by-interval fashion.
%
I.e., chains are split into intervals and the $(i + 1)$-th interval is checked after the $i$-th interval of all chains has been checked.
%
An invalid chain $\chain^*$ is removed out of the chain buffer once any interval of $\chain^*$ fails the $\mathsf{IsValidChain}$ verification.
%
And all valid chains are added to \parallelTrees as a fork (we do this interval-by-interval as well).
%
After filtering all invalid chains, the next step runs the chain selection procedure for every chain index in the current interval, and replaces each chain in \parallelChains with the longest fork in the corresponding tree in \parallelTrees.

\begin{cccAlgorithm}
    {$\mathsf{UpdateLocalChain}(\parallelChains, \parallelTrees, \mathcal{N}_0 = \{ \chain_1, \ldots, \chain_N \})$}
    {update-local-chain}
    {Parties filter invalid chains and select the heaviest one.}

    \begin{algorithmic}[1]
        \LineComment{This algorithm should only be called by fully-synchronized parties.}

        \For{$itvl$ \textbf{from} $1$ \textbf{to} \interval}
        \For{each $\chain \in \mathcal{N}_0$}
        \State{Let $i$ be the chain index of \chain}
        \If{$\mathsf{IsValidChain}(\chain, i, \parallelTrees, itvl) = \true$}
        \State{Add $\chainPrefixUB{\chain}{itvl \cdot \syncLen}$ to the $i$-th tree in \parallelTrees}
        \Else \Comment{invalid chain from $itvl$}
        \State{Remove \chain from $\mathcal{N}_0$}
        \EndIf
        \EndFor
        \EndFor

        \For{$i$ \textbf{from} $1$ \textbf{to} $m$}
        \State{Set $\chain_{\mathrm{max}} \gets \parallelChains_i$}
        \For{each $\chain \in \parallelTrees_i$}
        \oneLineIf{$\chainDiff{\chain} > \chainDiff{\chain_{\mathrm{max}}}$}{$\chain_{\mathrm{max}} \gets \chain$}
        \EndFor
        \State{Replace the $i$-th chain in \parallelChains by $\chain_{\mathrm{max}}$}
        \EndFor

        \State{\Return $\parallelChains, \parallelTrees$}
    \end{algorithmic}
\end{cccAlgorithm}

\paragraph{Mining procedure.}
%
Once a party \party has prepared all information and updated its state, she can run the core mining procedure in~\cref{protocol:mining-procedure}.
%
When \localTime reports in the output generation phase, \party will include the fresh input blocks and check if she succeeds in the input-block mining procedure.
%
At the onset of a BA invocation (when internal variable \val is set to $\bot$), \party starts to build her own input, starting from a coinbase transaction $\tx_\party^{\text{base-tx}}$ that contains her public key $\mathsf{pk}$ and she also signs this transaction.

Note that, for simplicity, we adopt two functions \blockify and $\mathsf{ValidTx}$ (cf.~\cite{C:BMTZ17}) that translates a sequence of transactions to the ledger state and verifies an incoming transaction w.r.t. a ledger state respectively, which we omit the details.

\begin{cccProtocol}
    {$\mathsf{MiningProcedure}(\party, \sid)$}
    {mining-procedure}
    {The mining procedure of $m$ parallel blocks/input-blocks.}

    \begin{algorithmic}[1]
        \LineComment{The following steps are executed in an (maintain-ledger, sid, minerID)-interruptible manner:}


        \State{$h \gets \varepsilon$}
        \Comment{Prepare chain head}

        \For{$i$ \textbf{from} $1$ \textbf{to} $m$}
        \State{Parse the last block of $i$-th chain in \parallelChainsLocal as \block}
        \Comment{Possibly genesis block (CRS)}
        \State{$h \gets h \concat \stringSegment{H(\block)}{i}{m}$}
        \EndFor


        \If{$\mathsf{isStageOG}(\localTime)$}

        \State{Set $h^\ast \gets \varepsilon$ and $st \gets \varepsilon$}

        \LineComment{Prepare fresh randomness}
        \For{$i$ \textbf{from} $1$ \textbf{to} $m$}
        \State{Parse the last block in the \textsf{VC} stage of $i$-th chain in \parallelChainsLocal as \block}
        \State{$h^\ast \gets h^\ast \concat \stringSegment{H(\block)}{i}{m}$}
        \EndFor

        \LineComment{Prepare block content}
        \State{$\mathbf{N} \gets \varepsilon$}
        \For{$i$ \textbf{from} $1$ \textbf{to} $m$}
        \State{Parse the $i$-th chain in \parallelChainsLocal as \chain}
        \State{Set $IB \gets \{\inputBlock' \in \inputBlockBuffer \mathbin| \mathsf{IsValidInputBlock}(\inputBlock', \chain, i, \parallelTreesLocal) = \true \}$}
        \State{Set $IB' \gets \{ \inputBlock' \in IB \mathbin| \inputBlock' \in \block' \in \chain \}$}
        \State{$\mathbf{N} \gets \mathbf{N} \concat (IB \backslash IB')$}
        \EndFor
        \State{$st \gets \mathsf{blockify}(\mathbf{N})$}

        \Else
        \State{Set $h^\ast \gets 0^\kappa$ and $st \gets 0^\kappa$}
        \EndIf

        \LineComment{Prepare chain reference $h'$}
        \State{Set $h' \gets \mathtt{snapshot}[\interval - 1]$}

        \LineComment{Prepare input \val}

        \If{$\val = \bot$}
        \State{Set $\buffer' \gets \buffer$, $\vec{N} \gets \tx^{\text{base-tx}}_\party(\mathsf{pk})$, and $\val \gets \blockify(\vec{N})$}
        \Repeat
        \State{Parse $\buffer'$ as sequence $(\tx_1, \ldots, \tx_n)$}
        \For{i \textbf{from} $1$ \textbf{to} $n$}
        \If{$\mathsf{ValidTx}(\tx_i, \state \concat \val) = 1$}
        \State{Set $\vec{N} \gets \vec{N} \concat \tx_i$}
        \State{Remove $\tx_i$ from $\buffer'$}
        \State{Set $\val \gets \blockify(\vec{N})$}
        \EndIf
        \EndFor
        \Until{$\vec{N}$ does not increase any more}
        \EndIf


        \LineComment{\mforone PoW mining}
        \State{$u \gets H(ctr, \localTime, h, st, h^\ast, h', \val)$}

        \State{Set $\mathsf{newChain}, \mathsf{newIB}$ as \false}

        \For{$i = 1$ \textbf{to} $m$}

        \State{Set $T_i \gets \mathsf{ExtractMiningTarget}(\chain, \epoch)$ where \chain is the $i$-th chain in \parallelChainsLocal}

        \If{$\stringSegment{u}{i}{m} < T_i$}
        \Comment{Extend $i$-th chain}
        \State{Set $\block \gets \langle ctr, \localTime, h, st, h^\ast, h', \val \rangle$ and $\mathsf{newChain} \gets \true$}
        \State{Append \block to the $i$-th chain of \parallelChainsLocal and \parallelTreesLocal}
        \EndIf

        \If{$\mathsf{isStageOG}(\localTime)$ \textbf{and} $\stringSegmentRev{u}{i}{m} < T_i$}
        \State{Set $\inputBlock \gets \langle ctr, \localTime, h, st, h^\ast, h', \val \rangle$ and $\mathsf{newIB} \gets \true$}
        \EndIf
        \EndFor

        \LineComment{Diffuse the extended chain and wait}
        \oneLineIf{$\mathsf{newChain} = \true$}{send $(\textsc{diffuse}, \sid, \parallelTreesLocal)$ to $\funcDiffuse^{\textsf{bc}}$ and set anchor here.\footnote{Upon next activation of this procedure, it resumes from the anchor set last time.}}

        \oneLineIf{$\mathsf{newIB} = \true$}{Send $(\textsc{diffuse}, \sid, \inputBlock)$ to $\funcDiffuse^{\textsf{input}}$ and set anchor here.}
        \State{Set $ctr \gets ctr + 1$, give up activation and set anchor here.}
    \end{algorithmic}
\end{cccProtocol}

\paragraph{Interval output algorithm.}
%
The following algorithm, on input parallel chains \parallelChains and a target interval $itvl$, outputs a triple $(\{ val_i \}_{i \in [m]}, \{ ref_i \}_{i \in [m]}, king)$.
%
The $i$-th element in the first vector $val_i$ is the output of the $i$-th chain in interval $itvl$ (i.e., the majority value of all input-blocks); and the $i$-th element in the second vector $ref_i$ is the reference to $i$-th chain in the previous interval (possibly being $\bot$); and $king$ is a single value extracted from the input block with minimum hash from the first chain.

Note that, each $ref_i$ is extracted by observing an invocation of weak agreement over the parallel chains in this interval.
%
I.e., we run $m$ weak agreement protocol instances in parallel, where each one follows \cref{protocol:approximate-consensus} in~\cref{subsec:apa-honst-majority}.

\begin{cccAlgorithm}
    {$\mathsf{ExtractIntervalOutput}(\parallelChains, itvl)$}
    {extract-interval-output}
    {Parties extract the output of a given interval.}

    \begin{algorithmic}[1]
        \State{Initialize $\{ val_i \}_{i \in [m]}, \{ ref_i \}_{i \in [m]}$ and $king$}

        \State{Initialize $m$ empty vectors $\{ \mathbf{R}_i \}_{i \in [m]}$}

        \For{$i$ \textbf{from} $1$ \textbf{to} $m$}
        \State{Initialize an empty vector $\mathbf{V}$ and $m$ empty vectors $\{ \mathbf{R}'_i \}_{i \in [m]}$}
        \For{each $\inputBlock \in \{\block \mathbin| \block \in \parallelChains_i \wedge \mathsf{isStageOG}(\block)\}$}
        \State{Parse \inputBlock as $\langle \cdot, \cdot, \cdot, \cdot, \cdot, h', val \rangle$}
        \State{Append $val$ to $\mathbf{V}$}
        \State{\textbf{For} $i$ \textbf{from} $1$ \textbf{to} $m$ \textbf{do} Append $\stringSegment{h'}{i}{m}$ to $\mathbf{R}'_i$}
        \EndFor

        \oneLineIf{$v'$ accounts for majority in $\mathbf{V}$}{$val_i \gets v'$}
        \State{\textbf{For} $i$ \textbf{from} $1$ \textbf{to} $m$ \textbf{do} Sort $\mathbf{R}'_i$ non-decreasingly then append $\med(\mathbf{R}'_i)$ to $\mathbf{R}_i$}

        \EndFor

        \For{$i$ \textbf{from} $1$ \textbf{to} $m$}
        \If{$\exists v$ that account for more than $3m/4$ elements in $\mathbf{R}_i$}
        \State{Set $ref_i \gets v$}
        \Else
        \State{Set $ref_i \gets \bot$}
        \EndIf
        \EndFor

        \State{Choose $\inputBlock^*$ s.t. $H(\inputBlock^*) = \min \{ H(\inputBlock) \mathbin| \inputBlock \in \block \in \parallelChains_1 \wedge \mathsf{isStageOG}(\block) \}$}
        \State{Set $king \gets \inputBlock^*.val$}

        \State{\Return $(\{ val_i \}_{i \in [m]}, \{ ref_i \}_{i \in [m]}, king)$}
    \end{algorithmic}
\end{cccAlgorithm}

\paragraph{State update procedure.}
%
We first present the basic state update algorithm in Chain-King Consensus.

\begin{cccProtocol}
    {$\mathsf{ChainKingUpdateState}(\party, \sid)$}
    {update-state-chain-king}
    {The state update procedure for Chain-King Consensus.}

    \begin{algorithmic}[1]
        \LineComment{This algorithm is called once in each interval.}
        \State{Set $(\{ val_i \}_{i \in [m]}, \cdot, king) \gets \mathsf{ExtractIntervalOutput}(\parallelChainsLocal, \interval)$}

        \State{Let $v$ denote the most frequent element in $\{ val_i \}_{i \in [m]}$ and $c$ its frequency}

        \If{$\interval \mod 3 = 1$}

        \oneLineIf{$c > m / 2$}{set $\val \gets v$}
        \oneLineIf{$c > 3m / 4$}{set $\decide \gets \true, \lock \gets \true$}

        \ElsIf{$\interval \mod 3 = 2$}

        \oneLineIf{$c > m / 2$}{set $\val \gets v$}
        \oneLineIf{$c > 3m / 4$}{set $\lock \gets \true$}

        \Else

        \oneLineIf{$\lock = \false$}{set $\val \gets king$}
        \oneLineIf{$\decide = \true$}{set $\state \gets \state \concat \val$ and $\val \gets \bot$}
        \oneLineIf{$\decide = \false$ \textbf{and} $\lock = \true$}{set $\lock \gets \false$}

        \EndIf
    \end{algorithmic}
\end{cccProtocol}

Note that in our SMR protocol, when a party \party is about to finish local interval \interval, she first stores her local view of the chains in this interval into $\mathtt{snapshot}$ and then run Chain-King Consensus with super-interval expansion and an additional lottery stage at the beginning which we detail below (cf.~\cref{subsec:new-smr-protocol}).
%
We omit the details on super-phase expansion and use $itvl$ to denote the super-interval index in a BA invocation.

\begin{cccProtocol}
    {$\mathsf{UpdateState}(\party, \sid)$}
    {update-state}
    {The state update procedure for \pSMR with fast fairness.}

    \begin{algorithmic}[1]
        \LineComment{This algorithm is called once in each interval.}

        \LineComment{Bookkeep local view of current interval}
        \State{Set $\mathtt{snapshot}[\interval] \gets \varepsilon$}
        \label{code:state-update-snapshot-start}
        \For{$i$ \textbf{from} $1$ \textbf{to} $m$}
        \State{Parse the $i$-th chain in \parallelChainsLocal as \chain}
        \State{Parse hash of last block on $\chainPrefixUB{\chain}{\interval \cdot \syncLen - \syncLen_{\mathsf{RC}}}$ as $h$}
        \State{$\mathtt{snapshot}[\interval] \gets \mathtt{snapshot}[\interval] \concat \stringSegment{h}{i}{m}$}
        \EndFor
        \label{code:state-update-snapshot-end}

        \LineComment{Update internal ledger states}
        \State{Let $\mathbf{V} = (v_1, \ldots, v_n)$ denote the output of $m$ chains respectively in super-interval $itvl$ and $king$ the minimum hash from the first chain}

        \State{Let $v$ denote the most frequent element in $\mathbf{V}$ and $c$ its frequency}

        \If{$\interval \mod 3 = 1$} \Comment{Lottery first.}

        \oneLineIf{$\lock = \false$}{set $\val \gets king$}
        \oneLineIf{$\decide = \true$}{set $\gets \state \concat \val$ and $\val \gets \bot$}
        \oneLineIf{$\decide = \false$ \textbf{and} $\lock = \true$}{set $\lock \gets \false$}


        \ElsIf{$\interval \mod 3 = 2$}

        \oneLineIf{$c > m / 2$}{set $\val \gets v$}
        \oneLineIf{$c > 3m / 4$}{set $\decide \gets \true, \lock \gets \true$}

        \Else

        \oneLineIf{$c > m / 2$}{set $\val \gets v$}
        \oneLineIf{$c > 3m / 4$}{set $\lock \gets \true$}

        \EndIf
    \end{algorithmic}
\end{cccProtocol}


\paragraph{Synchronization procedure.}
%
Parties call $\mathsf{SyncProcedure}$ when their local clock enters the last round in an interval and adjusts their clock by computing \shift based on their local parallel chains.
%
Note that the ``retorted'' timestamps are marked explicitly with the next interval index.
%
Thus, for each interval, this procedure is called only once.

\begin{cccProtocol}
    {$\mathsf{SyncProcedure}(\party, \sid)$}
    {sync-procedure}
    {Parties update their local clocks.}
    
    \begin{algorithmic}[1]
        \LineComment{Only called when: \party is alert, $\localTime = \protocolTime{\interval}{\interval \cdot \syncLen}$ and $\interval > 0$}

        \State{Initialize $\{ \mathtt{clockShift}_i \}_{i \in [m]}$ as an empty vector}
        \For{$i = 1$ \textbf{to} $m$}
        \State{Set \chain as $i$-th chain in \parallelChainsLocal}
        \State{$B \gets \{ \block \mathbin| (\block \in \chain) \wedge \mathsf{isStageOG}(\timestamp{\block}) = \true \}$}

        \State{$IB \gets \{\inputBlock \mathbin| (\inputBlock \in \block \in B) \wedge (\timestamp{\inputBlock} = \protocolTime{\interval}{\cdot}) \}$}

        \LineComment{Find representative beacon and compute recommendation.}

        \For{each $\inputBlock \in IB$}

        \State{Find unique $\inputBlock' \in \inputBlockBuffer$ s.t. $\inputBlock' = \inputBlock$. If inexistent, set $\inputBlock' \gets \bot$.}

        \If{$\inputBlock' \neq \bot$}
        \State{Set $\mathsf{arrivalTime}(\inputBlock) \gets \mathsf{arrivalTime}(\inputBlock')$}

        \State{$\textsf{recom}(\inputBlock, i) \gets \timestamp{\inputBlock} - \mathsf{arrivalTime}(\inputBlock)$}
        \Else
        \State{$IB \gets IB \mathbin \backslash \{ \inputBlock \}$}
        \EndIf
        \EndFor

        \State{$\mathtt{clockShift}_i \gets \med \{ \textsf{recom}(\inputBlock, i) \mathbin| \inputBlock \in IB \}$}
        \EndFor

        \LineComment{Compute interval shift using \cref{eq:sync-shift}}
        \State{$\shift_{\interval} \gets \mathsf{avg}(\mathsf{select}(\mathsf{reduce}(\mathtt{clockShift}, \eta), \eta))$}

        \LineComment{Update beacon registry}
        \For{each \inputBlock with $\mathsf{arrivalTime}(\inputBlock) = (a, \mathsf{temp})$}
        \State{$\mathsf{arrivalTime}(\inputBlock) \gets (a + \shift_{\interval}, \mathsf{final})$}
        \EndFor

        \LineComment{Update local time}
        \State{Set $\localTime \gets (\interval + 1, \round + \shift_{\interval})$}

        \State{Let $\mathcal{N}_0$ be the subsequence of \futureChains s.t. $\chain \in N_0 : \Leftrightarrow \forall \block \in \chain : \timestamp{\block} < \localTime$}

        \State{Remove each $\chain \in \mathcal{N}_0$ from \futureChains}

        \State{Call $\textsf{updateLocalChain}(\parallelChainsLocal, \parallelTreesLocal, \mathcal{N}_0)$ to update \parallelChainsLocal and \parallelTreesLocal}

        \State{Send $(\textsc{diffuse}, \sid, \parallelTreesLocal)$ to $\funcDiffuse^{\textsf{bc}}$ and proceed from here upon next activation of this procedure}
    \end{algorithmic}

    \textsc{Output:} The protocol outputs \ok to its caller (but not to \Z).
\end{cccProtocol}


\paragraph{Finishing a round.}
%
Once a party \party has done its actions in a round, \party claims finishing current round by calling $\mathsf{FinishRound}$ and sending \textsc{clock-update} to \funcDriftingClock.

\begin{cccProtocol}
    {$\mathsf{FinishRound}(\party, \sid)$}
    {round-finish}
    {Finishing a round.}

    \begin{algorithmic}[1]
        \While{A $(\textsc{clock-update}, \Z)$ has not been received during the current round}
        \State{Give up activation (set the anchor here)}
        \EndWhile
        \State{Send $(\textsc{clock-update}, \sid_C)$ to \funcDriftingClock.}
    \end{algorithmic}
\end{cccProtocol}

\paragraph{The joining procedure.}
%
Honest yet un-synchronized parties run the $\mathsf{JoiningProcedure}$ to synchronize their internal state (i.e., their local clock and blockchain state).
%
Parties run this procedure for constantly many rounds, by passively listen to the protocol execution, bootstrap the blockchain, keep track of the input-block local arrival time, and then adjust their local clock based on these information.

\begin{cccProtocol}
    {$\mathsf{JoiningProcedure}(\party, \sid)$}
    {joining-procedure}
    {The joining procedure for a fresh new party.}

    \begin{algorithmic}[1]
        \LineComment{Phase A: state-reset}
        \State{Call $\mathsf{UpdateLocalTime}(\party, \sid)$} \Comment{Align with newest round}
        \If{$\localTime > \protocolTime{1}{1}$}
        \State{$\localTime \gets \protocolTime{1}{1}$}
        \State{$\fetchCompleted \gets \false, \futureChains \gets \emptyset, \inputBlockBuffer \gets \emptyset, \buffer \gets \emptyset$}
        \State{Set input-block arrival timetable as empty array}
        \EndIf

        \LineComment{Phase B: chain-convergence}
        \While{$\localTime < \protocolTime{1}{1} + t_{\mathsf{off}}$}
        \If{$\fetchCompleted = \false$}
        \State{Call $\mathsf{FetchInformation}(\party, \sid)$ and denote the fetched chains by $\mathcal{N} = (\chain_1, \ldots , \chain_N)$}
        \State{Call $\mathsf{UpdateLocalChain}(\parallelChainsLocal, \parallelTreesLocal, \mathcal{N})$ to update \parallelChainsLocal and \parallelTreesLocal}
        \State{$\fetchCompleted \gets \true$}
        \State{Call $\mathsf{FinishRound}(\party, \sid)$}
        \EndIf
        \State{Call $\mathsf{UpdateLocalTime}(\party, \sid)$ to update \localTime}
        \EndWhile

        \LineComment{Phase C: input-block gathering}
        \While{$\localTime < \protocolTime{1}{1} + t_{\mathsf{off}} + t_{\mathsf{gather}}$}

        \If{$\fetchCompleted = \false$}

        \State{Call $\mathsf{FetchInformation}(\party, \sid)$ and denote output by $(\chain_1, \ldots, \chain_N)$, $(\tx_1, \ldots, \tx_k)$}
        \State{Set $\buffer \gets \buffer \concat (\tx_1, \ldots, \tx_k)$}
        \State{Set $\futureChains \gets \futureChains \concat (\chain_1, \ldots, \chain_N)$}
        \State{Call $\mathsf{ProcessInputBlocks}(\party, \sid, \emptyset)$ and mark all arrival time with $\mathsf{temp}$}
        \State{Call $\mathsf{UpdateLocalChain}(\parallelChainsLocal, \parallelTreesLocal, \futureChains)$ to update \parallelChainsLocal and \parallelTreesLocal}

        \State{Let \parallelChains denotel the paralle chains after pruning \kbootstr blocks on all chains in \parallelChainsLocal}
        \State{Set $\protocolTime{itvl^*}{\cdot} \gets \med \{ \timestamp{\block} \mathbin| \block~\text{is the tip block of}~\chain \wedge \chain \in \parallelChains \}$}

        \oneLineIf{$\mathtt{snapshot}[itvl^*] = \bot$}{set $\mathtt{snapshot}[itvl^*]$ using \cref{protocol:update-state} Line~\ref*{code:state-update-snapshot-start} to \ref*{code:state-update-snapshot-end}}

        \State{$\fetchCompleted \gets \true$}
        \State{Call $\mathsf{FinishRound}(\party, \sid)$}

        \EndIf

        \State{Call $\mathsf{UpdateLocalTime}(\party, \sid)$ to update \localTime}

        \EndWhile

        \LineComment{Phase D: synchronization and state-update}

        \State{Set $i$ as the second minimum positive integer s.t. $\mathtt{snapshot}[i] \neq \bot$}

        \While{$\mathtt{snapshot}[i] \neq \bot$}
        \State{Initialize $\{ \mathtt{clockShift}_j \}_{j \in [m]}$ as an empty vector}
        \For{$j$ \textbf{from} $1$ \textbf{to} $m$}
        \State{Set \chain as $j$-th chain in \parallelChainsLocal}
        \State{$B \gets \{ \block \mathbin| (\block \in \chain) \wedge \mathsf{isStageOG}(\timestamp{\block}) = \true \}$}

        \State{$IB \gets \{\inputBlock \mathbin| (\inputBlock \in \block \in B) \wedge (\timestamp{\inputBlock} = \protocolTime{i}{\cdot}) \}$}

        \LineComment{Find representative beacon and compute recommendation.}

        \For{each $\inputBlock \in IB$}

        \State{Find unique $\inputBlock' \in \inputBlockBuffer$ s.t. $\inputBlock' = \inputBlock$. If inexistent, set $\inputBlock' \gets \bot$.}

        \If{$\inputBlock' \neq \bot$}

        \State{Set $\mathsf{arrivalTime}(\inputBlock) \gets \mathsf{arrivalTime}(\inputBlock')$}

        \State{$\textsf{recom}(\inputBlock, j) \gets \timestamp{\inputBlock} - \mathsf{arrivalTime}(\inputBlock)$}
        \Else
        \State{$IB \gets IB \mathbin \backslash \{ \inputBlock \}$}
        \EndIf
        \EndFor

        \State{$\mathtt{clockShift}_i \gets \med \{ \textsf{recom}(\inputBlock, j) \mathbin| \inputBlock \in IB \}$}
        \EndFor

        \LineComment{Compute interval shift using \cref{eq:sync-shift}}
        \State{$\shift_i \gets \mathsf{avg}(\mathsf{select}(\mathsf{reduce}(\mathtt{clockShift})))$}

        \For{each \inputBlock with $\mathsf{arrivalTime}(\inputBlock) = (a, \mathsf{temp})$}
        \State{Set $\mathsf{arrivalTime}(\inputBlock) \gets (a + \shift_i, \mathsf{temp})$}
        \EndFor

        \State{Set $\localTime \gets \localTime + \shift_i$}

        \State{Set $i \gets i + 1$}
        \EndWhile

        \State{Set $\isSync \gets \true$ and $t_{\mathsf{work}} \gets \localTime - 1$}
        \State{Run \textsf{UpdateLocalChain} to filter chains with future timestamps}

        \For{each beacon $\inputBlock \in \inputBlockBuffer$ with $\timestamp{\inputBlockBuffer} \le (i + 1) \cdot \syncLen$}
        \State{Parse $\mathsf{arrivalTime}(\inputBlock)$ as $(a, \mathsf{temp})$ and define $\mathsf{arrivalTime}(\inputBlock) = (a, \mathsf{final})$}
        \EndFor
    \end{algorithmic}

    \textsc{Output:} The protocol outputs \ok to its caller (but not to \Z)
\end{cccProtocol}
