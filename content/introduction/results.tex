\subsection{Summary of Our Results}
\label{subsec:our-results}

In this paper we put forth a protocol for SMR in the permissionless setting that, for the first time, satisfies all the properties enumerated above, while ensuring in addition that Liveness is achieved in expected-constant time (a condition usually referred to as ``fast [transaction] settlement''\footnote{We will use `symbols' and `transactions' interchangeably, as it is common in the blockchain literature.}).
%
See \cref{table:PoW-BA} for a comparison with existing results.

\begin{table*}[ht]
    \newcommand*{\checkMark}{\ding{51}}
    \newcommand*{\crossMark}{{\color{lightgray} \ding{55}}}

    \begin{tabularx}{\textwidth}{Y @{} Y @{} Y @{} Y @{} Y}
        \toprule
        \textbf{Protocol}
         & \textbf{Settlement}
         & \textbf{Dynamic}
         & \textbf{Timekeeping}
         & \textbf{Drifting Clocks}
        \\ \midrule
        \cite{EC:GarKiaLeo15}
         & $\bigO(\polylog \kappa)$
         & \crossMark
         & \crossMark
         & \crossMark
        \\ \midrule
        \cite{C:GarKiaLeo17}
         & $\bigO(\polylog \kappa)$
         & \checkMark
         & \crossMark
         & \crossMark
        \\ \midrule
        \cite{CCS:BKTFV19}
         & Expected $\bigO(1)^\ast$
         & \crossMark
         & \crossMark
         & \crossMark
        \\ \midrule
        \cite{TCC:FGKR20}
         & Expected $\bigO(1)^\ast$
         & \checkMark
         & \crossMark
         & \crossMark
        \\ \midrule
        \cite{EC:GarKiaShe24}
         & Expected $\bigO(1)$
         & \crossMark
         & \crossMark
         & \crossMark
        \\ \midrule
        \cite{TCC:GarKiaShe22}
         & $\bigO(\polylog \kappa)$
         & \checkMark
         & \checkMark
         & \crossMark
        \\ \midrule
        This work
         & Expected $\bigO(1)$
         & \checkMark
         & \checkMark
         & \checkMark
        \\ \bottomrule
        
        \caption{Comparison of permissionless SMR protocols in terms of their settlement time, ability to handle dynamic participation and timekeeping against drifting clocks. $^\ast$: Running time here refers to the optimistic case when no transactions are ``double-spent.''}
        \label{table:PoW-BA}
    \end{tabularx}
\end{table*}

As the table shows, our permissionless SMR protocol is the first to be ``self-sufficient'' in terms of being its own time keeper, and in the more realistic model where parties' clocks may drift.
%
In contrast, all existing protocols either have assumed the availability of a global clock; or that such functionality is realized by means of some heuristic, as in the case of Bitcoin~\cite{EPRINT:GarKiaLeo20}; or that parties' clocks might be off but proceed at the same speed~\cite{TCC:GarKiaShe22}.

We achieve this by means of a \textbf{new clock synchronization} protocol for the permissionless setting that achieves a constant (i.e., asymptotically optimal) ``skew'' tolerating an adversary that controls less than $50\%$ of the computational power (a bound we prove to be optimal in our setting), and may be of independent interest.
%
Specifically, the skew of the clocks is $\bigTheta(\clockDrift \delay)$, where \clockDrift is the upper bound on clock drift rates and \delay is the maximum network latency.
%
Importantly, and in contrast to previous treatments of network delay in the blockchain literature (e.g.,~\cite{EC:PasSeeshe17,C:BMTZ17,EPRINT:GarKiaLeo20,EPRINT:CEMMPS20}), where the delay is specified in terms of network `rounds,' here it is specified in terms of nominal time, as in the original distributed-computing formulation~\cite{JACM:DwoLynSto88}, as in our setting the notion of `round' is a local/per-party one, and, further, may not be of uniform duration (in nominal time).
%
(Refer to~\cref{subsec:drifting-clocks-and-synchronization} for background on fault-tolerant clock synchronization basics.)

Furthermore, the protocol allows for a fully transient participation pattern---i.e., even if every protocol participant fully functions in only one round and immediately goes offline, our protocol still remains secure---and is based on the following key components:
%
(i) A novel blockchain approach that utilizes \emph{parallel blockchains};
%
(ii) a new clock adjustment algorithm that runs \emph{approximate agreement} on top of parallel blockchains and enables parties to concentrate a clock shift value so they can adjust their local clocks maintaining a bounded overall skew and near optimal accuracy (up to an arbitrary small constant) with respect to nominal time;
%
and (iii) a new bootstrapping protocol that allows a newly joining party with no knowledge other than the initial setup (i.e., the genesis block) to ``catch up'' with the synchronized parties.
%
I.e., after bootstrapping, a new party adjusts its local clock to exhibit a  deviation from the existing online parties' clocks that is well-bounded (in the order of the network latency).

Our parallel blockchain construction allows, in addition, a state update for every interval of constant duration, thus making it possible to run a PoW-based expected-constant-time consensus protocol (specifically, the ``Chain-King Consensus'' protocol in~\cite{EC:GarKiaShe24}) ``on top'' of our clock synchronization protocol, to yield a distributed ledger whose security does not rely on parties having access to a global clock, and where \textbf{all} incoming symbols (e.g., transactions) can be added to parties' logs in expected-constant time.
%
Furthermore, our protocol utilizes a suitable pre-agreement phase that facilitates Fast Fairness.
