\section{Introduction}
\label{sec:introduction}

In the \emph{state machine replication} (SMR) problem~\cite{CSUR:Schneider90}, a state machine $\Pi = (Q, \Sigma, \delta)$, with $\delta (q, \sigma) \rightarrow q'$ its transition function, is to be replicated across a number of parties.
%
At any given time, the machine $\Pi$ is at a certain state that results from the application of a sequence of symbols $\sigma_1 \ldots \sigma_n$ to its transition function.
%
In the most common interpretation of the problem, it is mandated that each party \party maintains a log, denoted by $\mathsf{log}_\party$, of such symbols such that the following two fundamental properties are satisfied:
%
\begin{cccItemize}[noitemsep]
    \item \textbf{Consistency:}  At any two times $t,t'$, for any two parties $\party, \party'$, it holds that $\mathsf{log}_\party \not \preceq \mathsf{log}_{\party'}$ implies $ \mathsf{log}_{\party'} \preceq \mathsf{log}_\party $, where  $\preceq$ stands for the prefix relation.

    \item \textbf{Liveness:} If all honest parties attempt to introduce a symbol $\sigma$ into the log, then after $u$ steps in nominal time it holds that every honest party's log will include  $\sigma$.
\end{cccItemize}

We note that the liveness property, as expressed above, refers to a notion of \emph{nominal} time, which will not be available to the parties --- indeed, to achieve it, the parties themselves may have to synchronize with each other in order to approximate nominal time using their local clocks which might be drifting:
%
In particular, the setting of interest here is being equipped with local clocks that run at different speeds but where there is an upper bound \clockDrift on how fast they can run compared to nominal time.

To properly reflect this timing requirement into our objectives, we will focus on state machines that incorporate time into their state.
%
Specifically, such a state machine has the following structure $\Pi = (Q \times \mathbb{N}, \Sigma, \delta)$ and includes a special ``clock tick'' symbol $\sigma_\mathsf{ct}$ that satisfies $\delta((q,t), \sigma_\mathsf{ct})= (q,t + 1)$, for any $q, t$.
%
Let \round be the resulting SMR clock value and, with foresight, denote by $\varGamma$ its accuracy (cf.~\cref{subsec:drifting-clocks-and-synchronization}).
%
The relevant property regarding time keeping is as follows:
%
\begin{cccItemize}[noitemsep]
    \item \textbf{$\varGamma$-Timekeeping:}
    %
    Let $\varGamma \in \mathbb{R}^+$.
    %
    At nominal time $t$, the state machine internal clock \round satisfies $(1+ \varGamma)^{-1} \leq  \round / t \leq (1+\varGamma) $.
    %
    We refer to \round as the ``SMR time.''
\end{cccItemize}
%
Sometimes we will refer to an SMR that keeps its own time under the mere assumption that parties have drifting local clocks as ``self-sufficient.''

Contrary to the ``classical'' SMR setting where parties have point-to-point channels and/or are capable of authenticating digital signatures issued by each other, we are interested in a setting where parties are \textbf{unacquainted}, in the sense that they have no information about each other, and communicate via an unauthenticated ``diffusion channel'' that allows the adversary to inject any arbitrary number of messages.
%
Moreover, parties may come and go without any announcement to others whatsoever, with their overall number fluctuating over time; hence, the common protocol design technique of counting the number of incoming messages may not be particularly helpful.
%
This is what sometimes is referred to as the ``permissionless'' setting.

The permissionless setting suggests a rather insurmountable target for SMR as it can be easily seen by the following argument:
%
A set $A$ of $n$ parties is about to engage in the protocol, while another set $A'$ of equal number of parties are about to do the same --- parties in $A$ are unaware of the existence of $A'$ and vice versa.
%
Consider now two symbols $\sigma \neq \sigma'$ so that parties in $A$ attempt to insert $\sigma$ while parties in $A'$ attempt to insert $\sigma'$.
%
The adversary can expedite message delivery within the sets $A, A'$ and hence  eventually $A$, (resp., $A'$) will have to settle $\sigma$ (resp., $\sigma'$), due to Liveness.
%
It follows that, if we are to solve SMR in this setting, we have to assume that communication delays are always bounded, say, by a quantity \delay.
%
Even with bounded delays though, an adaptation of the classical FLP impossibility result \cite{JACM:FisLynPat85} can be used to show that SMR is still impossible (cf.~\cite{CSF:PasShi17}).
%
The challenge stems from the ability of the adversary to simulate protocol messages ``in its head'' without any cost and the fact that lack of authentication enables the adversary to serve such messages as if they were part of a legitimate protocol execution.
%
To circumvent this impossibility, we will assume the existence of an (random) oracle $H(\cdot)$ and impose a restriction on how many queries to $H(\cdot)$ the adversary can pose.
%
This small asymmetry between honest and adversarial parties turns out to be sufficient to circumvent this impossibility.

The Liveness property introduced above is preconditioned on all honest parties attempting to insert the same symbol in the log.
%
In the permissioned setting, where all parties are acquainted with each other, it can be straightforward to create circumstances where honest parties attempt to insert the same symbol (e.g., all parties may propose symbols in a round-robin fashion).
%
In the permissionless setting, however, this is problematic:
%
Without any form of authentication or throttling, it is possible for an adversary to flood the incoming tapes with symbols it prefers while stifling useful inputs.
%
To address this consideration we will require an additional \emph{fairness} property for permissionless SMRs:
%
Any honest party gets a chance to introduce a special symbol of their choice\footnote{Think of a `coinbase' transaction in Bitcoin.} with probability
in proportion to the number of queries posed by the party to the  oracle $H(\cdot)$.
%
More formally:
%
\begin{cccItemize}[noitemsep]
    \item \textbf{Fairness:}
    %
    Let \round denote the SMR time when a special symbol $\sigma^*$ is introduced to the log of all honest parties.
    %
    Then, for the window of nominal time $W_\round$ ending in \round of length $B\in \mathbb{N}$, a property parameter, we say that party \party has weight $w_\party$ that is equal to the number of queries posed by \party to $H(\cdot)$ during $W_\round$.
    %
    Fairness requires that special symbol $\sigma^*$ is issued from an honest party \party with probability $q$ such that  $q \ge (1 - \epsilon) (w_\party / \sum_{\party'} w_{\party'} )$, for any $\epsilon\in (0, 1)$.
\end{cccItemize}
%
A parameter of particular interest of the above property is the length of the time window $W_\round$; i.e., how fast does the protocol sample the distribution of queries to $H(\cdot)$ to allow a new symbol in the log.
%
In particular, if the length of this window is of constant size, we will refer to the property as \emph{Fast Fairness}.

\subsection{Summary of Our Results}
\label{subsec:our-results}

In this paper we put forth a protocol for SMR in the permissionless setting that, for the first time, satisfies all the properties enumerated above, while ensuring in addition that Liveness is achieved in expected-constant time (a condition usually referred to as ``fast [transaction] settlement''\footnote{We will use `symbols' and `transactions' interchangeably, as it is common in the blockchain literature.}).
%
See \cref{table:PoW-BA} for a comparison with existing results.

\begin{table*}[ht]
    \newcommand*{\checkMark}{\ding{51}}
    \newcommand*{\crossMark}{{\color{lightgray} \ding{55}}}

    \begin{tabularx}{\textwidth}{Y @{} Y @{} Y @{} Y @{} Y}
        \toprule
        \textbf{Protocol}
         & \textbf{Settlement}
         & \textbf{Dynamic}
         & \textbf{Timekeeping}
         & \textbf{Drifting Clocks}
        \\ \midrule
        \cite{EC:GarKiaLeo15}
         & $\bigO(\polylog \kappa)$
         & \crossMark
         & \crossMark
         & \crossMark
        \\ \midrule
        \cite{C:GarKiaLeo17}
         & $\bigO(\polylog \kappa)$
         & \checkMark
         & \crossMark
         & \crossMark
        \\ \midrule
        \cite{CCS:BKTFV19}
         & Expected $\bigO(1)^\ast$
         & \crossMark
         & \crossMark
         & \crossMark
        \\ \midrule
        \cite{TCC:FGKR20}
         & Expected $\bigO(1)^\ast$
         & \checkMark
         & \crossMark
         & \crossMark
        \\ \midrule
        \cite{EC:GarKiaShe24}
         & Expected $\bigO(1)$
         & \crossMark
         & \crossMark
         & \crossMark
        \\ \midrule
        \cite{TCC:GarKiaShe22}
         & $\bigO(\polylog \kappa)$
         & \checkMark
         & \checkMark
         & \crossMark
        \\ \midrule
        This work
         & Expected $\bigO(1)$
         & \checkMark
         & \checkMark
         & \checkMark
        \\ \bottomrule
        
        \caption{Comparison of permissionless SMR protocols in terms of their settlement time, ability to handle dynamic participation and timekeeping against drifting clocks. $^\ast$: Running time here refers to the optimistic case when no transactions are ``double-spent.''}
        \label{table:PoW-BA}
    \end{tabularx}
\end{table*}

As the table shows, our permissionless SMR protocol is the first to be ``self-sufficient'' in terms of being its own time keeper, and in the more realistic model where parties' clocks may drift.
%
In contrast, all existing protocols either have assumed the availability of a global clock; or that such functionality is realized by means of some heuristic, as in the case of Bitcoin~\cite{EPRINT:GarKiaLeo20}; or that parties' clocks might be off but proceed at the same speed~\cite{TCC:GarKiaShe22}.

We achieve this by means of a \textbf{new clock synchronization} protocol for the permissionless setting that achieves a constant (i.e., asymptotically optimal) ``skew'' tolerating an adversary that controls less than $50\%$ of the computational power (a bound we prove to be optimal in our setting), and may be of independent interest.
%
Specifically, the skew of the clocks is $\bigTheta(\clockDrift \delay)$, where \clockDrift is the upper bound on clock drift rates and \delay is the maximum network latency.
%
Importantly, and in contrast to previous treatments of network delay in the blockchain literature (e.g.,~\cite{EC:PasSeeshe17,C:BMTZ17,EPRINT:GarKiaLeo20,EPRINT:CEMMPS20}), where the delay is specified in terms of network `rounds,' here it is specified in terms of nominal time, as in the original distributed-computing formulation~\cite{JACM:DwoLynSto88}, as in our setting the notion of `round' is a local/per-party one, and, further, may not be of uniform duration (in nominal time).
%
(Refer to~\cref{subsec:drifting-clocks-and-synchronization} for background on fault-tolerant clock synchronization basics.)

Furthermore, the protocol allows for a fully transient participation pattern---i.e., even if every protocol participant fully functions in only one round and immediately goes offline, our protocol still remains secure---and is based on the following key components:
%
(i) A novel blockchain approach that utilizes \emph{parallel blockchains};
%
(ii) a new clock adjustment algorithm that runs \emph{approximate agreement} on top of parallel blockchains and enables parties to concentrate a clock shift value so they can adjust their local clocks maintaining a bounded overall skew and near optimal accuracy (up to an arbitrary small constant) with respect to nominal time;
%
and (iii) a new bootstrapping protocol that allows a newly joining party with no knowledge other than the initial setup (i.e., the genesis block) to ``catch up'' with the synchronized parties.
%
I.e., after bootstrapping, a new party adjusts its local clock to exhibit a  deviation from the existing online parties' clocks that is well-bounded (in the order of the network latency).

Our parallel blockchain construction allows, in addition, a state update for every interval of constant duration, thus making it possible to run a PoW-based expected-constant-time consensus protocol (specifically, the ``Chain-King Consensus'' protocol in~\cite{EC:GarKiaShe24}) ``on top'' of our clock synchronization protocol, to yield a distributed ledger whose security does not rely on parties having access to a global clock, and where \textbf{all} incoming symbols (e.g., transactions) can be added to parties' logs in expected-constant time.
%
Furthermore, our protocol utilizes a suitable pre-agreement phase that facilitates Fast Fairness.

\subsection{Related Work}
\label{subsec:related-work}

\paragraph{SMR protocols.}
%
In the traditional, permissioned setting~\cite{CSUR:Schneider90}, an SMR protocol is executed by a fixed set of servers that are ``acquainted'' with each other (e.g., they know each other's public key and/or have explicit point-to-point communication channels with each other).
%
Numerous protocols have been proposed in this setting, with more recent work focusing on efficient constructions (e.g.,~\cite{OSDI:CasLis99,PODC:YMRGA19,EC:ChaPasShi19,SP:AMNRY20,EUROSys:DKSS22}), invariably exploiting the ability of participants to issue votes in the form of signatures and have those votes counted by the recipients to ensure that a suitable quorum has been reached and parties can settle the transactions in the log.
%
It is clear that such techniques are not readily amenable to the permissionless setting.

The only known design technique for achieving SMR in the permissionless setting with dynamic participation is based on the Bitcoin blockchain (cf.~\cite{Nak08,C:GarKiaLeo17}).
%
In this protocol, the oracle $H(\cdot)$ is utilized to realize a proof-of-work (PoW) functionality \cite{C:DwoNao92} with a moderate difficulty that is periodically adjusted to accommodate fluctuations in participation.
%
From our perspective, the protocol exhibits a number of deficiencies, namely, Liveness with a parameter proportional to the security parameter, lack of fairness (due to block withholding selfish mining attacks \cite{FC:EyaSir14,EC:GarKiaLeo15}) and Timekeeping whose accuracy is based on the participants' having access to a shared global clock (so the protocol is not self-sufficient).
%
Follow up work to Bitcoin addressed some of these issues individually; for example in the static participant setting, Fruitchains focused on the issue of fairness~\cite{PODC:PasShi17}, and Prism on the issue of transaction throughput~\cite{CCS:BKTFV19}.
%
For further overview as well as impossibility results see~\cite{CSF:PasShi17,RSA:GarKia20}.
%
Nonetheless, to date, no SMR protocol in the permissionless setting has been proposed that addresses all the relevant considerations simultaneously fairness and self-sufficiency.

\paragraph{Clock synchronization.}
%
The clock synchronization problem has been studied for over four decades by the distributed computing community.
%
\emph{Synchronizers} are distributed fault-tolerant protocols that solve the synchronization problem --- to mention a few, pulse synchronizer \cite{JACM:LamMel85,PODC:LunLyn84,PODC:HSSD84,JACM:SriTou87,PODC:LenLos22} where parties re-synchronize their clocks periodically and one-shot synchronizer \cite{InfCon:LunLyn84,JoComp:HalMegMun85} where the goal is to synchronize clocks with initial large skews.
%
These traditional protocols operate in the permissioned model where the participants are known \emph{a priori} (or, parties can join upon approval from all honest parties, cf.~\cite{PODC:HSSD84}).

Dolev \textit{et al.} \cite{JCSS:DHS86} showed that without setup assumptions, clock synchronization cannot be achieved with more than one-third of the corrupted parties (i.e., it requires $t < n / 3$).
%
With unforgeable signatures, the corruption bound can be improved to $t < n / 2$ \cite{JACM:SriTou87,PODC:LenLos22}, or to the dishonest majority setting~\cite{PODC:HSSD84}.
%
When the protocol allows new parties to join, a majority of the honest parties is necessary.

\emph{Bounded skew} (i.e., the level of simultaneity) is a fundamental property when measuring the performance of synchronizers.
%
In the fault-free setting, Lynch and Welch~\cite{InfCon:LunLyn84} showed that even if clocks run at exactly the same rate, network uncertainty is impossible to overcome.
%
Precisely, in a network with \delay delay (again, measured in real time) and $n$ processors, it is impossible to synchronize clocks more closely than $\delay (1 - 1 / n)$.
%
This result was later extended to any network by \cite{JoComp:HalMegMun85}.
%
Since this result holds under strong assumptions, they also apply to the drifting clock model.
%
When it comes to pulse synchronizers, the interval between two synchronization points should be at least $\bigTheta(\delay)$ rounds apart from each other, during which the clock has already drifted for more than $\bigTheta(\clockDrift \delay)$ time.
%
Hence, $\bigTheta(\clockDrift \delay)$ turns out to be the (asymptotically) optimal skew one could expect.

With pulse synchronization, another fundamental metric is the degree of deviation from real time, namely \emph{accuracy}.
%
Dolev \textit{et al.} showed that synchronization is a non-trivial task only when the target logical time stays in a linear envelope of real time \cite{JCSS:DHS86}.
%
Srikanth and Toueg \cite{JACM:SriTou87} showed that the logical linear envelope cannot be smaller than the physical one; and, that to stay in the same envelope with physical clocks, a majority of the participants should be honest.

More recently, the clock synchronization problem has been re-considered in the context of blockchains, however in a \textbf{weaker model} of imperfect local clocks where the adversary can (only) apply an \textbf{additive} drift $\varPhi_{\mathsf{clock}}$ to the honest parties' clocks throughout the whole execution~\cite{EC:BGKRZ21,TCC:GarKiaShe22}.
%
In the ``classical'' setting, this model is not meaningful in that when parties are always online, their skew will never deviate further than $\varPhi_{\mathsf{clock}}$.
%
The above works considered this weaker model in the permissionless environment with dynamic participation where parties can join and leave as they please.
%
As a result, protocol participants can no longer filter messages based on the total number of parties, and newly joining parties should be able to bootstrap and synchronize their clocks with honest parties by passively listening to the network.
%
Since the clock model is weaker, those results in~\cite{EC:BGKRZ21,TCC:GarKiaShe22} are not directly comparable to the traditional literature with drifting clocks.
%
From our perspective, the key observation is that there are two challenging dimensions to clock synchronization: drifting clocks, analyzed mostly in the permissioned literature, and imperfect clocks exhibiting a bounded skew in a setting where participants join and leave the protocol at will.
%
Handling both these dimensions at once is a critical missing piece for achieving SMR in the permissionless setting.

\paragraph{Timing models in cryptography.}
%
In distributed computing parlance, following the treatment from~\cite{JACM:DwoLynSto88}, the synchrony hierarchy yields three levels:
%
(i) Synchronous --- there are \emph{known} upper bounds on clock drift and maximum network delay.
%
(ii) Partially-synchronous\footnote{\cite{JACM:DwoLynSto88} also considers a second type of partially synchrony where there is an \emph{unknown} global stable time (GST) such that a \emph{known} maximum network delay holds after GST however no restriction on message transmission is imposed before GST.} --- upper bounds on clock drift and delay do exist yet they are \emph{unknown} to honest parties;
%
note that in~\cite{JACM:DwoLynSto88}, delay is measured by a \emph{real time clock} outside the system.
%
(iii) Asynchronous --- there is no upper bound on local clock speeds and message transmission (though messages between honest parties are eventually delivered).

Timing shows as a tool in cryptography, nonetheless, there lacks a unified approach on modeling time and delays, especially they are ill-defined within the UC setting.
%
Here we provide a short survey.

Timing models came into consideration for the secure concurrent zero-knowledge protocols \cite{STOC:DwoNaoSah98,C:DwoSah98}, where Dwork, Naor and Sahai proposed the $(\alpha, \beta)$-constraint (for some [known] $\alpha \le \beta$) --- for any two parties \party and  $\party'$, if \party measures $\alpha$ elapsed time on its local clock and $\party'$, starting after \party, measures $\beta$ elapsed time on its local clock, then $\party'$ finishes after \party --- an assumption that is implicit under the (appropriately-bounded) drifting clock model.
%
In concurrent composition of secure computation \cite{STOC:KalLinPra05}, Kalai, Lindell and Prabhakaran work in a model where local clocks run within known bounded rates and message transmission takes up to a known \delay time.
%
Yet, they define delay based on \emph{all clocks} --- i.e., maximum message transmission is subject to the bound on clock drifts (cf.~\cite[footnote 10]{STOC:KalLinPra05}).

In the UC setting, despite its inherent asynchronous message transmission scheduling, Katz \textit{et al.} \cite{TCC:KMTZ13} model synchronous computation via the co-design of a synchronized clock and bounded-delay channel functionalities.
%
In more detail, real time (represented as a global round counter) is forwarded only when all honest parties claim finishing their computation in that round, and message is delivered to a party after receiving sufficiently many \textsc{fetch} requests where honest parties issue one \textsc{fetch} query per \emph{local} round, and the adversary can accumulatively increment the fetch counter for up to \delay rounds.
%
Recently, Canetti \textit{et al.} \cite{CSF:CHMV17} provide a treatment of the network time protocol by means of a global clock functionality with bounded additive drifts and unbounded response delays.
%
Aided with such global clock, they also present another functionality that measures the local clock drifts (i.e., relative time elapsed between two global clock reads).

\paragraph{Parallel blockchains and \mforone PoWs.}
%
Parallel blockchain designs have been found applications in improving the performance of blockchain-based SMR.
%
For example, in ``Ledger Combiners'' \cite{TCC:FGKR20} a ranking function is proposed on top of a set of  parallel chains to accelerate transaction settlement --- for the case of non-conflicting transactions.
%
In~\cite{EC:GarKiaShe24}, parallel blockchains serve as a platform to port classical consensus protocols and enable building a PoW-based, expected-constant-time Byzantine agreement protocol, and achieve SMR with expected-constant settlement time, albeit for the \emph{static} participation case.

The fundamental cryptographic primitive that secures parallel blockchains is \emph{\mforone Proof-of-Work}\footnote{Pronounced ``m-for-1'' PoW.}, which guarantees that the mining procedure of single chains are mutually independent (or, sub-independent with bounded statistical distance --- cf.~\cite{TCC:FGKR20}); i.e., the adversary cannot gain advantage on a specific chain by dropping from others.
%
The \mforone PoW is in fact a generalization of the \twoforone PoW technique introduced in~\cite{EC:GarKiaLeo15} that achieves an equitable distribution of inputs contributed into a blockchain based on the oracle queries posed by the participants to the oracle $H(\cdot)$.
%
In~\cite{EC:GarKiaLeo15}, the \twoforone PoW primitive was used to improve the corruption resiliency of permissionless Byzantine agreement, and has also been utilized in~\cite{PODC:PasShi17} to design a blockchain protocol in the \emph{static} participation setting that offers a notion of fairness
(although not fast).
%
Combining fairness with dynamic participation and clock synchronization (i.e., self-provided timekeeping) has remained until now an open question.

In~\cite{EC:GarKiaShe24}, the \mforone PoW scheme that yields SMR with expected-constant settlement time works by running $m = \bigTheta(\polylog \kappa)$ chains in parallel, where $\kappa$ is the output length of the hash function, with $\bigOmega(\polylog \kappa)$ bits allocated to each chain.
%
As mentioned above, this protocol is designed for the static participation setting.
%
Finally, note that in the PoS context, full independence among $m$ parallel chains can be achieved by separately evaluating $m$ VRFs with different nonces, which yields a simple but equivalent alternative construction to \mforone PoW.

\paragraph{On transient faults and dynamic participation.}
%
To wrap up, our setting is typified by a fluctuating number of participants who may come and go without announcement.
%
As a result, at nominal time $t$ the number of participants is $n_t$, with the initial number of parties being $n_0$.
%
Dynamic availability has been considered in prior work~\cite{AC:PasShi17,CCS:BGKRZ18}, but not in the completely unacquainted setting as we do here:
%
These works operated under the assumption that a consistent public-key directory is known to all participants and the adversary may choose an arbitrary subset of registered parties to run the protocol.
%
In the permissionless setting, dynamic availability was considered in~\cite{C:GarKiaLeo17} under the assumption that the sequence $n_1,n_2, \ldots$ is \textbf{not} adaptively determined --- an important restriction that is not present in our modeling.


\paragraph{Organization of the paper.}
%
The rest of the paper is organized as follows.
%
In the next section we provide a technical overview of our results.
%
In~\cref{sec:preliminaries}, we introduce our clock, network and adversary models and provide basic notation and definitions.
%
Then, in~\cref{sec:reaching-weak-approx-agreement}, we solve two ``one-shot problems'' --- permissionless Weak Agreement and permissionless Approximate Agreement --- as basic building blocks.
%
In~\cref{sec:permissionless-smr} we present the full permissionless SMR protocol, which is based on the new parallel blockchain construction (\cref{subsec:new-parallel-blockchain}) and permissionless clock synchronization procedure (\cref{subsec:clock-sync-procedure}).
%
The full protocol analysis is presented in~\cref{sec:full-protocol-analysis}.

