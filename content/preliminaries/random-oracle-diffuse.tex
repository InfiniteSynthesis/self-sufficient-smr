\subsection{Random Oracle, Network and Adversarial Model}
\label{subsec:ro-network-corruption}

\paragraph{Random oracle.}
%
By convention, we model the cryptographic hash function $H$ with output in $\{0, 1\}^\kappa$ (which is used to generate proofs of work [PoWs]) as a random oracle \funcRO~\cite{CCS:BelRog93}.

\begin{cccFunctionality}{\funcRO}
    {random-oracle}
    {The random oracle.}

    The functionality is parameterized by a security parameter $\kappa$.

    \addtocounter{table}{-1}
    \begin{tabularx}{.9\textwidth}{c  X}
        \toprule[.3mm]
        \textbf{State Variable}
         & \textbf{Description}
        \\ \midrule[.3mm]
        $\partyset \gets \emptyset$
         & The set of registered parties.
        \\ \midrule
        $H \gets \emptyset$
         & A dynamically updatable function table where $H[x] = \bot$ denotes the fact that no pair of the form $(x, \cdot)$ is in $H$.
        \\ \bottomrule[.3mm]
    \end{tabularx}

    \begin{cccItemize}[noitemsep]
        \item \textbf{Eval.} Upon receiving $(\textsc{eval}, \sid, x)$ from some party $\party \in \partyset$ (or from \adv on behalf of a corrupted \party), do the following:
        %
        \begin{cccEnum}[nosep]
            \item If $H[x] = \bot$ sample a value $y$ uniformly at random from $\{0, 1\}^\kappa$ and set $H[x] \gets y$.
            \item Return $(\textsc{eval}, \sid, x, H[x])$ to the requestor.
        \end{cccEnum}
    \end{cccItemize}
\end{cccFunctionality}


We express our honest majority condition in terms of computational power, measured in particular by the number of queries to the RO that the parties are allowed to make per \emph{nominal time-step}, as opposed to expressing it by the number of parties (i.e., the ``flat'' model where parties are assumed to have equal computational power---cf.~\cite{EC:GarKiaLeo15}).

\begin{definition}[Honest Majority]
    Let $h_r, t_r$ denote the number of \emph{alert} and \emph{non-alert} random oracle queries at \emph{nominal} time $r$, respectively.
    %
    Then, for all $r \in \mathbb{N}$, it holds that $h_r > t_r$.
\end{definition}

This restriction on the number of RO queries is captured by a wrapper functionality on \funcRO via counting the number of \emph{alert} and other queries (see below) per nominal-time step.
%
The adversary is allowed to dynamically and adaptively determine the number of alert random oracle queries per nominal-time step, as long as it does not violate the restrictions imposed by the $(\gamma, s)$-respecting environment (see~\cref{def:respecting-environment} in the sequel).

\begin{cccFunctionality}
    {\wrapper{\funcRO}}
    {random-oracle-wrapper}
    {The random oracle wrapper.}

    This functionality maintains state variables as follows.

    \addtocounter{table}{-1}
    \begin{tabularx}{.9\textwidth}{c  X}
        \toprule[.3mm]
        \textbf{State Variable}
         & \textbf{Description}
        \\ \midrule[.3mm]
        $\partyset \gets \emptyset$
         & The set of registered parties; the current set of corrupted parties is denoted by $\partyset'$.
        \\ \midrule
        $\tau \gets 0$
         & The (real-time) clock tick counter.
        \\ \midrule
        $h_\tau$
         & An upper bound which restricts the \func-evaluations of all alert parties at time $\tau$.
        \\ \midrule
        $q_\honestPartySet, q_\adv \gets 0$
         & The alert/adversary evaluation counter.
        \\ \bottomrule[.3mm]
    \end{tabularx}

    \paragraph{Pre-mining attack handling (executed only if $\tau = 0$):}
    %
    \begin{cccItemize}[nosep]
        \item Upon receiving $(\textsc{eval}, \sid, x)$ from \adv on behalf of a corrupted party $P \in \partyset'$, forward the request to \funcRO and return to \adv whatever \funcRO returns.

        \item Upon receiving $(\textsc{Retrieved}, \sid)$ from \funcCRS, set $\tau = 1$.
    \end{cccItemize}

    \paragraph{Relaying inputs to the random oracle:}
    %
    \begin{cccItemize}[nosep]
        \item Upon receiving $(\textsc{eval}, \sid, x)$ from \adv on behalf of a corrupted party $P \in \partyset'$ or a de-synchronized party \party, first execute \textit{Round Reset}, then do the following.
        %
        \begin{cccEnum}[nosep]
            \item Set $q_\adv \gets q_\adv + 1$.
            \item \textbf{If} $q_\adv \le h_\tau$ \textbf{then} forward the request to \funcRO and return to \adv whatever \funcRO returns.
        \end{cccEnum}

        \item Upon receiving $(\textsc{eval}, \sid, x)$ from an alert party \party, first execute \textit{Round Reset}, then do the following.
        %
        \begin{cccEnum}[nosep]
            \item Set $q_\honestPartySet \gets q_\honestPartySet + 1$.
            \item \textbf{If} $q_\honestPartySet \le h_\tau$ \textbf{then} forward the request to \funcRO and return to \party whatever \funcRO returns.
            \item \textbf{If} $q_\honestPartySet \ge h_\tau$ \textbf{then} send $(\textsc{clock-update}, \sid_C)$ to \funcDriftingClock.
        \end{cccEnum}
    \end{cccItemize}

    \paragraph{Corruption handling:}
    %
    \begin{cccItemize}[nosep]
        \item Upon receiving $(\textsc{corrupt}, \sid, \party)$ from the adversary, set $\partyset' \gets \partyset' \cup \party$.
    \end{cccItemize}

    \medskip\emph{Procedure Round-Reset:}
    %
    Send $(\textsc{clock-read}, \sid_C)$ to \funcDriftingClock and receive $(\textsc{clock-read}, \allowbreak \sid_C, \tau')$ from \funcDriftingClock. If $|\tau - \tau' | > 0$, then do the following.
    %[
    \begin{cccEnum}[nosep]
        \item Set $q_\honestPartySet, q_\adv \gets 0$ and $\tau \gets \tau'$.

        \item Send $(\textsc{next-round})$ to \adv and receive as response $(\textsc{next-round}, h^*_{\tau'})$.
        %
        \textbf{If} $(h_1, h_2, \ldots, h'_{\tau'})$ is $(\gamma, s)$-respecting (\cref{def:respecting-environment}) \textbf{then} set $h_{\tau'} = h'_{\tau'}$; \textbf{else} set $h_{\tau'} = h_{\tau' - 1}$.
    \end{cccEnum}
\end{cccFunctionality}


In addition, pre-mining attack prevention is captured by restricting the number of adversarial queries after a fresh CRS is released (which we model as \funcCRS).

\begin{cccFunctionality}
    {\funcCRS}
    {CRS}
    {The common reference string.}

    The functionality is parameterized by a distribution $\mathcal{D}$.

    \begin{cccItemize}[noitemsep]
        \item \textbf{Retrieve.} Upon receiving $(\textsc{Retrieve}, \sid)$ from some party \party (or from \adv on behalf of a corrupted \party), do the following:
        %
        \begin{cccEnum}[nosep]
            \item If activated for the first time, choose a value $d \gets \mathcal{D}$, and send $(\textsc{Retrieved}, \sid)$ to \wrapper{\funcRO} and \wrapper{\funcDiffuse}.

            \item Return $(\textsc{Retrieve}, d)$ to \party.
        \end{cccEnum}
    \end{cccItemize}
\end{cccFunctionality}

\paragraph{The bounded-delay network.}
%
Regarding communication amongst parties, we consider a peer-to-peer diffusion network, where the message dissemination has an (unknown) \delay-bounded delay.
%
In more detail, an honest message sent at time $t$ will be received by all other honest parties before time $t + \delay$; regarding messages sent by the adversary, if $t$ is the earliest time such that at least one honest party receives those messages, they are guaranteed to be delivered to all honest parties before time $t + \delay$ (i.e., honest parties keep ``echoing'' messages).

We capture this communication network with \funcDiffuse (\cref{functionality:diffuse}).
%
Recall that existing diffuse functionalities (cf.~\cite{C:BMTZ17}) model delays in the following manner:
%
There is a fetch counter per message per recipient such that when each time an honest party \party is activated and received a new tick from the clock, \party fetches on \funcDiffuse which reduces counters by $1$ for all messages delivering to \party; she then receives a subset of those messages with counters reset to $0$.
%
Regarding the adversary, he can increase the counter for each message and recipient for up to \delay in accumulation (as well as swapping the order of messages).

By convention, different types of messages are diffused by different functionalities, and we write $\funcDiffuse^{\textsf{bc}}$, $\funcDiffuse^{\textsf{input}}$, $\funcDiffuse^{\textsf{tx}}$ to denote the network for chains, input blocks and transactions.

\begin{cccFunctionality}
      {\funcDiffuse}
      {diffuse}
      {The diffusion network.}

      \newcommand*{\msgid}{\ensuremath{\mathsf{mid}}\xspace}
      \newcommand*{\vecM}{\ensuremath{\vec{M}}\xspace}

      This functionality maintains state variables as follows.

      \begin{minipage}{\linewidth}
            \addtocounter{table}{-1}
            \begin{tabularx}{.9\textwidth}{c  X}
                  \toprule[.3mm]
                  \textbf{State Variable}
                   & \textbf{Description}
                  \\ \midrule[.3mm]
                  $\delay \gets 0$
                   & The maximum network latency.
                  \\ \midrule
                  $\partyset \gets \emptyset$
                   & The set of registered parties.
                  \\ \midrule
                  $\vecM \gets [] $
                   & A dynamically updatable list of quadruples $(m, \msgid, D_{\msgid}, \party)$ where $D_{\msgid}$ denotes the fetch counter.
                  \\ \bottomrule[.3mm]
            \end{tabularx}
      \end{minipage}

      \medskip
      \paragraph{Setting the delay:}
      %
      \begin{cccItemize}[nosep]
            \item Upon  receiving $(\textsc{set-delay}, \sid, d)$ from the adversary \adv, \textbf{if} \textsc{set-delay} has never been received \textbf{then} set $\delay = d$. Return $(\textsc{set-delay}, \sid, \ok)$ to \adv.
      \end{cccItemize}

      \paragraph{Network capabilities:}
      %
      \begin{cccItemize}[nosep]
            \item Upon receiving $(\textsc{diffuse}, \sid, m)$ from some
            $\party_s \in \partyset$, where $\partyset = \{ \party_1, \ldots, \party_n \}$ denotes the current party set, do:
            %
            \begin{cccEnum}[nosep]
                  \item Choose $n$ new unique message-IDs $\msgid_1, \ldots, \msgid_n$.
                  \item Initialize $2n$ new variables $D_{\msgid_1} := D^{MAX}_{\msgid_1} \ldots := D_{\msgid_n} := D^{MAX}_{\msgid_n} := 1$ and a per message-delay $\delay_{\msgid_i} = \delay$ for $i \in [n]$.
                  \item Set  $\vecM := \vecM \concat (m, \msgid_1, D_{\msgid_1}, \party_1) \concat \ldots \concat (m, \msgid_n, D_{\msgid_n}, \party_n)$.
                  \item Send $(\textsc{diffuse}, \sid, m, \party_s, (\party_1, \msgid_1), \ldots ,(\party_n, \msgid_n))$ to the adversary.
            \end{cccEnum}

            \item Upon receiving $(\textsc{fetch}, \sid)$ from $\party \in \partyset$, or from \adv on behalf of a corrupted party \party:
            %
            \begin{cccEnum}[nosep]
                  \item For all tuples $(m, \msgid, D_\msgid, \party) \in \vecM$, set $D_\msgid := D_\msgid - 1$.
                  \item Let $\vecM^\party_0$ denote the subvector \vecM including all tuples of the form $(m, \msgid, D_\msgid, \party)$ with $D_\msgid \le 0$ (in the same order as they appear in \vecM).
                  %
                  Delete all entries in $\vecM^\party_0$ from \vecM and in case some $(m, \msgid, D_\msgid, \party)$ is in
                  $\vecM^\party_0$, where \party is honest, set $\delay_{\msgid'} = \delay$ for any $(m, \msgid', D_{\msgid'}, (\cdot, \sid))$ in \vecM and replace this record by $(m, \msgid', \min \{ D_{\msgid'}, \delay \}, \party')$.
                  \item Output $\vecM^\party_0$ to \party (if \party is corrupted, send $\vecM^\party_0$ to \adv).
            \end{cccEnum}
      \end{cccItemize}

      \paragraph{Additional adversarial capabilities:}
      %
      \begin{cccItemize}[nosep]
            \item Upon receiving $(\textsc{diffuse}, \sid, m)$ from some corrupted $\party_s \in \partyset$ (or from \adv on behalf of $\party_s$ if corrupted),
            % do 
            execute it the same way as an honest-sender diffuse, with the only difference that $\delay_{\msgid_i} = \infty$.

            \item Upon receiving $(\textsc{delays}, \sid, (T_{\msgid_{i_1}}, \msgid_{i_1}), \ldots, (T_{\msgid_{i_\ell}}, \msgid_{i_\ell}))$ from the adversary do the following for each pair $(T_{\msgid_{i_j}}, \msgid_{i_j})$:
            %
            if $D^{MAX}_{\msgid_{i_j}} + T_{\msgid_{i_j}} \le \delay_{\msgid_{i_j}}$ and $\msgid_{i_j}$ is a message-ID of receiver $\party = (\cdot, \sid)$ registered in the current \vecM, set $D_{\msgid_{i_j}} := D_{\msgid_{i_j}} + T_{\msgid_{i_j}}$ and set $D^{MAX}_{\msgid_{i_j}} := D^{MAX}_{\msgid_{i_j}} + T_{\msgid_{i_j}}$; otherwise, ignore this pair.

            \item Upon receiving $(\textsc{swap}, \sid, \msgid, \msgid')$ from the adversary, if \msgid and $\msgid'$ are message-IDs registered in the current \vecM, then swap the triples $(m, \msgid, D_\msgid, (\cdot, \sid))$ and $(m, \msgid', D_{\msgid'}, (\cdot, \sid))$ in \vecM.
            %
            Return $(\textsc{swap}, \sid)$ to the adversary.

            \item Upon receiving $(\textsc{get-reg}, \sid)$ from \adv, return the response $(\textsc{get-reg}, \sid, \partyset)$ to \adv.
      \end{cccItemize}
\end{cccFunctionality}

We highlight that such mechanism does \emph{not} work in our drifting clock model with \funcDriftingClock (it only works with a global clock where parties proceed with the same speed).
%
This is because, when modeling delays via fetches on the diffuse functionality, honest parties that experience relatively fast local rounds would request more \textsc{fetch} commands than the slow ones in the same window of nominal time --- i.e., if an honestly-sent message is set the same delay for two parties then it delivers to the fast one earlier; yet our goal is to model delays measured in the perspective of the nominal time (regardless of parties' local understanding of time).

We resolve this issue by introducing a new wrapper on \funcDiffuse (\cref{functionality:wrapper-diffuse}) that restricts the adversary's capability to delay messages for up to \delay nominal time.
%
After registering on \funcDriftingClock to learn the nominal time ticks, the wrapper functionality dynamically relays parties' \textsc{fetch} request to \funcDiffuse so that in every nominal-time step, exactly one fetch operation is relayed to \funcDiffuse for each honest party (even if that honest party receives multiple ticks from \funcDriftingClock).
%
Meanwhile, for each party \party that proceeds slowly and may not activate in a given nominal time, the wrapper queries \funcDiffuse on behalf of \party, buffers the response messages and delivers them to \party upon the next time \party interacts with the wrapper.

\begin{cccFunctionality}
    {\wrapper{\funcDiffuse}}
    {wrapper-diffuse}
    {The wrapper of diffuse network.}

    This functionality maintains state variables as follows.

    \addtocounter{table}{-1}
    \begin{tabularx}{.9\textwidth}{c  X}
        \toprule[.3mm]
        \textbf{State Variable}
         & \textbf{Description}
        \\ \midrule[.3mm]
        $\partyset \gets \emptyset$
         & The set of registered parties; the current set of corrupted parties is denoted by $\partyset'$.
        \\ \midrule
        $\tau \gets 0$
         & The (real-time) clock tick counter.
        \\ \midrule
        $\mathsf{fetch}(\party, \tau) \gets 0$
         & The fetch variable for party \party at nominal time $\tau$.
        \\ \midrule
        $\mathsf{buffer}_\party \gets []$
         & The fetch buffer for party \party.
        \\ \bottomrule[.3mm]
    \end{tabularx}

    \paragraph{Relaying inputs to the diffuse network:}
    %
    \begin{cccItemize}[nosep]
        \item Upon receiving $(\textsc{diffuse}, \sid, m)$ from \adv on behalf of some corrupted $\party \in \partyset'$, parse $m$ as blocks $\block_1, \ldots, \block_n$.
        %
        For each $\block_i$, if $\block_i$ has not been queried to \funcRO, send $(\textsc{eval}, \sid, \block_i)$ from a corrupted party.

        \item Upon receiving $(\textsc{fetch}, \sid)$ from an honest party \party, if $\mathsf{fetch}(\party, \tau) = 1$ ignore this request.
        %
        Otherwise, execute the following:
        %
        \begin{cccEnum}[nosep]
            \item Forward $(\textsc{fetch}, \sid)$ to \funcDiffuse and recevie as response $\vec{M}$, return $\mathsf{buffer}_\party \concat \vec{M}$.
            \item Set $\mathsf{buffer}_\party \gets []$ and $\mathsf{fetch}(\party, \tau) \gets 1$.
        \end{cccEnum}
    \end{cccItemize}

    \paragraph{Corruption handling:}
    %
    \begin{cccItemize}[nosep]
        \item Upon receiving $(\textsc{corrupt}, \sid, \party)$ from the adversary, set $\partyset' \gets \partyset' \cup \party$.
    \end{cccItemize}

    \medskip\emph{Procedure Round-Reset:}
    %
    Send $(\textsc{clock-read}, \sid_C)$ to \funcDriftingClock and receive $(\textsc{clock-read}, \allowbreak \sid_C, \tau')$ from \funcDriftingClock. If $|\tau - \tau' | > 0$, then do the following.
    %[
    \begin{cccEnum}[nosep]
        \item Set $\tau \gets \tau'$.

        \item  For each honest party $\party$ such that $\mathsf{fetch}(\party, \tau) = 0$, send $(\textsc{fetch}, \sid)$ to \funcDiffuse from \party and recevie as response $\vec{M}$, set $\mathsf{buffer}_\party \gets \mathsf{buffer}_\party \concat \vec{M}$.
        %
        For each honest party $\party$ such that $\mathsf{fetch}(\party, \tau) = 1$, set $\mathsf{fetch}(\party, \tau) \gets 0$.

        \item Send $(\textsc{clock-update}, \sid_C)$ to \funcDriftingClock.
    \end{cccEnum}
\end{cccFunctionality}


\paragraph{Dynamic availability and respecting environment.}
%
In order to apply a more fine-grained classification on protocol participants, we follow the treatment in~\cite{CCS:BGKRZ18} and classify parties into different types based on their accessible resources and synchronization states.
%
Specifically, a party is (i) \emph{operational} if she is registered with the random oracle \funcRO, and \emph{stalled} otherwise; (ii) \emph{online} if she is registered with the network \funcDiffuse, and \emph{offline} otherwise; (iii) \emph{time-aware} if she is registered with the drifting clock \funcDriftingClock, and \emph{time-unaware} otherwise; and (iv) \emph{synchronized} if she has been participated in the protocol for sufficiently long time and held ``synchronized state'' and ``synchronized time'' with other synchronized parties, and \emph{desynchronized} otherwise.

We define \emph{alert} parties based on the classification above.
%
Specifically, alert parties are those who have access to all the resources and are synchronized.
%
They are the core set of parties to carry out the protocol.

Next, we define a ``respecting environment'' in terms of the computational power (cf.~\cite{TCC:GarKiaShe22}) as opposed to number of parties (cf.~\cite{C:GarKiaLeo17,EPRINT:GarKiaLeo20}).
%
Our honest-majority assumption is that during the whole protocol execution, the alert computational power is higher than the adversarial one.
%
We restrict the environment so that the number of such queries will be bounded in a certain fashion.

\begin{definition} \label{def:respecting-environment}
    For $\gamma \in \mathbb{R}^+$ we call the sequence $(h_r)_{r \in [0, B)}$, where $B \in \mathbb{N}$, $(\gamma, s)$-respecting if for any set $S \subseteq [0, B)$ of at most $s$ consecutive integers, $\max_{r \in S} h_r \le \gamma \cdot \min_{r \in S} h_r$.
\end{definition}
