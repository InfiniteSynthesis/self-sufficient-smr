\subsection{Weak/Approximate Agreement with Optimal Corruption Threshold}
\label{subsec:apa-honst-majority}

\cref{protocol:one-third-approximate-consensus} works only when the adversarial computational power is bounded by one-third, and in terms of AA the protocol offers concentration quality $\epsilon = 1 / 2$.
%
In this section we present a new protocol that improves the corruption threshold to honest majority for both Weak and Approximate Agreement.
%
Moreover, for Approximate Agreement, it achieves $\epsilon$-Agreement for an arbitrarily small $\epsilon > 0$ after a single invocation.

We first provide some intuition on why~\cref{protocol:one-third-approximate-consensus} fails in an honest majority setting, taking Approximate Agreement as an example.
%
Notice that with stronger adversarial computational power, the intersection of PoW message arrays held by an honest party at the decision phase becomes too small to account for two-thirds of its PoW message array.
%
Thus, since the common subset is not large enough, the \textsf{reduce} and mid-point computations can no longer guarantee $\epsilon$-Agreement for $\epsilon = 1 / 2$ and Validity at the same time.
%
In more detail, when the adversarial messages (and inconsistent honest messages due to network delay) can account for more than one-third of the messages, parties need to trim more values from both sides of the ordered array; otherwise, Validity would not hold because the mid-point computation may take some adversarially proposed values as input.
%
However, trimming more than one-third of the elements from both sides hurts $\epsilon$-Agreement, as the inequalities in~\cref{lemma:one-third-apa} no longer stand.

To circumvent the above situation, our solution is to categorize the different PoW messages based on their indices (in the $m$ parallel procedures) where they succeed, and select one message in each procedure to form a message array \textbf{V} of fixed size of $m$ values, thus providing a more refined approach to decide on an output value.

\paragraph{Protocol description.}
%
Here we show how to extend \cref{protocol:one-third-approximate-consensus} to the honest majority setting.
%
The protocol takes two additional parameters $k$ and $\eta$, which are explained below, and runs $m$ PoW-based Byzantine agreement (BA) procedures in parallel (cf.~\cite{EC:GarKiaLeo15})\footnote{We note that, while running a single invocation of such BA procedure \cite{EC:GarKiaLeo15} for $\bigTheta(\log^2 \kappa)$ rounds yields a full agreement, terminating after constantly many rounds does not give any form of weak agreement with overwhelming probability.}.
%
In each procedure, it builds a PoW-based blockchain (which at a high level follows the Nakamoto protocol) and binds the mining procedure of the chain with ``input-blocks'' using \twoforone PoWs (thus, each party maintains $m$ parallel blockchains which we denote by \parallelChains).
%
Specifically, for a RO query output $u$ and each chain index $i \in [m]$, alongside with checking if a block that is used to extend the chain is produced by evaluating $\stringSegment{u}{i}{m} < T$, the \twoforone mining further evaluates the reverse of this chunk \stringSegmentRev{u}{i}{m}.
%
If $\stringSegmentRev{u}{i}{m} < T$, an input-block \inputBlock containing the miner's input is mined and \inputBlock can be included in the corresponding $i$-th blockchain.
%
Yet, a single RO query can produce different valid blocks to extend different blockchains, as well as an input block that can be valid on multiple chains.
%
Refer to~\cref{protocol:approximate-consensus} Line~\ref*{code:mforone-pow-mining-begin} to \ref*{code:mforone-pow-mining-end} to see how the \mforone and \twoforone mining procedures are bound together in order to get $m$ independent parallel instances of the PoW-based BA protocol.
%
Also note that, parties extend each blockchain \chain independent following the longest chain selection rule (as specified in the Bitcoin backbone protocol~\cite{EC:GarKiaLeo15} which we omit the details on chain validation and selection here), and keeps including valid and unique input-blocks with respect to \chain.

After $\ell$ rounds, for each parallel chain $\chain_i$, parties extract all input-blocks in its prefix by pruning blocks based on their timestamp, subject to the ``common prefix'' parameter $k$.\footnote{In ordinary blockchains, after pruning $k$ blocks, parties hold a chain that is a prefix of any other party's with overwhelming probability, hence the term ``common prefix.'' However, since our protocol terminates in constant time, this pruning operation will lead to a common prefix only with constant probability.}
%
Then, these input-blocks are ordered based on their value contained and the median one is picked as the output of chain $\chain_i$.
%
This forms an array $\mathbf{V}$ of size $m$.
%
For Approximate Agreement, each party then decides its output based on its local $\mathbf{V}$ and a parameter $\eta < m / 2$, by first ordering $\mathbf{V}$ and removing the smallest and largest $\eta$ values, selecting the first element of every $\eta$ elements, and computing the average of the selected ones.
%
I.e., they output $\textsf{avg}(\mathsf{select}(\mathsf{reduce}(\mathsf{V}, \eta), \eta))$ and terminate.
%
In terms of Weak Agreement, parties output a value that accounts for a super-majority of the chains if it exists, and $\mathcal{\bot}$ otherwise.

\begin{cccProtocol}
    {$\mathsf{{\frac{1}{2}-}ApproxAgr}(\ell, m, T, k, \eta)$}
    {approximate-consensus}
    {Honest-majority Weak/Approxmate Agreeement.}

    \begin{algorithmic}[1]
        \LineComment{The following code is executed in each round \round (\val is the party input)}

        \If{$r \le \ell$}
        \State{Fetch incoming chains $(\chain_1, \ldots, \chain_N)$}

        \State{$\parallelChainsLocal \gets \mathsf{UpdateLocalChain}(\parallelChainsLocal, \chain_1, \ldots, \chain_N)$}
        \Comment{Select longest valid chains for each index}

        \State{Fetch incoming input-blocks $(\inputBlock_1, \ldots, \inputBlock_k)$}
        \State{Add valid $\inputBlock_1, \ldots, \inputBlock_{k'}$ to \buffer}

        \LineComment{\mforone PoW mining}

        \State{$h \gets \varepsilon$, $st \gets \varepsilon$} \label{code:mforone-pow-mining-begin}
        \For{$i = 1$ \textbf{to} $m$}
        \State{$h \gets h \concat \stringSegment{H(\chainHead{\parallelChains_i})}{i}{m}$}
        \State{Let $\buffer_i$ denote all \inputBlock that are valid yet not included w.r.t. $\parallelChains_i$}

        \State{$st \gets st \concat \blockify(\buffer_i)$\footnote{\blockify translates a sequence of transactions to the ledger state (cf.~\cite{C:BMTZ17}).}}

        \EndFor
        \State{$u \gets H(ctr, r, h, st, val)$}

        \For{$i = 1$ \textbf{to} $m$}
        \Comment{Check if PoW succeeds on any type of block.}
        \oneLineIf{$\stringSegment{u}{i}{m} < T$}{set $\parallelChains_i \gets \parallelChains_i \concat  \langle ctr, r, h, st, val \rangle$ and diffuse $\parallelChains_i$} \Comment{Extend chain}
        \oneLineIf{$\stringSegmentRev{u}{i}{m} < T$}{diffuse  $\inputBlock = \langle ctr, r, h, st, val \rangle$}
        \EndFor
        \State{$ctr \gets ctr + 1$} \label{code:mforone-pow-mining-end}

        \Else

        \State{Initialize $\mathbf{V}$ to an empty array}

        \For{$i$ \textbf{from} $1$ \textbf{to} $m$}
        \Comment{Extract output from parallel chains}

        \State{Initialize $\mathbf{M}$ to an empty array}
        \For{$\inputBlock \in \block \in \chainPrefixUB{\parallelChains_i}{\ell - k}$}
        \State{Parse \inputBlock as $\langle \cdot, \cdot, \cdot, \cdot, val \rangle$ and add $val$ to $\mathbf{M}$}
        \EndFor
        \State{Sort $\mathbf{M}$ and add $\med (\mathbf{M})$ to $\mathbf{V}$}
        \EndFor

        \LineComment{Decide final output}
        \If{deciding output for weak agreement}
        \State{Order $\mathbf{V}$ non-decreasingly and denote $n \gets |\mathbf{V}|$}
        \oneLineIf{$\mathbf{V}_\eta = \mathbf{V}_{n - \eta}$}{output $\mathbf{V}_{\eta}$ \textbf{else} output $\bot$}
        \ElsIf{deciding output for approximate agreement}
        \State{Output $\textsf{avg}(\mathsf{select}(\mathsf{reduce}(\mathbf{V}, \eta), \eta))$}
        \EndIf
        \EndIf
    \end{algorithmic}
\end{cccProtocol}


Next, in the following lemma we show that running the $m$ parallel chains and terminating in a constant number of rounds in a bounded delay network yields good properties on a fraction of the chains.
%
The two properties in~\cref{lemma:apa} have been proven for a synchronous network in~\cite{EC:GarKiaShe24}.
%
Here we extend this result to the bounded-delay network setting.

\begin{lemma} \label{lemma:apa}
    There exist parameterizations of~\cref{protocol:approximate-consensus} such that the following holds.
    %
    Let \parallelChains and $\parallelChains'$ denote the parallel chains held by two honest parties \party and $\party'$ at the end of round $\ell$, respectively.
    %
    There exists a subset $S \subseteq \{1, 2, \ldots, m \}$, $|S| > m - \eta$, such that for all $i \in S$, the following properties hold on chains $\chain = \parallelChains_i$ and $\chain' = \parallelChains'_i$:
    %
    \begin{enumerate}[label=\FlatSteel, leftmargin=*, nosep]
        \item \emph{\bf Agreement}: $\chainPrefixUB{\chain}{\ell - k} = \chainPrefixUB{\chain'}{\ell - k}$.

        \item \emph{\bf Honest input-block majority}: More than half of the input-blocks included in $\chainPrefixUB{\chain}{\ell - k}$ and $\chainPrefixUB{\chain'}{\ell - k}$ are produced by honest parties.
    \end{enumerate}
\end{lemma}

\begin{proof}[Proof (sketch)]
    Consider the $i$-th single chain $\chain = \parallelChains_i$ of a party \party.
    %
    There exist parameterizations such that at the end of round $\ell$, with constant probability $p > (m - \eta) / m$, $\chainPrefixUB{\chain}{\ell - k}$ yields a common view with any other honest party and includes a majority of honest input blocks.
    %
    Proving the above claim is mainly a reminiscence of the proof of~\cite[Theorem 2]{EC:GarKiaShe24}, with additionally extending it to the bounded-delay network (see, e.g., \cite[Section 7]{EPRINT:GarKiaLeo14}, for more details).

    Since any two single chains are mutually independent from each other, the probability that these two % good 
    properties hold on more than $p \cdot m  > m - \eta$ chains can be computed by the Chernoff bound (\cref{thm:chernoff-bounds}), with error probability negligibly small in the security parameter $\kappa$ (recall that $m = \polylog \kappa$).
\end{proof}

Given that the `good' properties hold on a sufficiently large subset of the values in the output decision array, we conclude that \cref{protocol:approximate-consensus} solves the Approximate Agreement problem with an arbitrarily small $\epsilon$.

\begin{theorem} \label{thm:honest-majority-approx-agreement}
    There exist protocol parameterizations such that \cref{protocol:approximate-consensus} is an $\epsilon$-secure Approximate Agreement protocol, for an arbitrarily small $\epsilon > 0$, against an adversary that controls less than half of the computational power, and all honest parties terminate at the end of round $\ell$, except with probability negligible in $\kappa$.
\end{theorem}

\begin{proof}
    Let $\eta = [\epsilon/(1 + 2\epsilon)]m$ and consider the arrays of size $m$ held by honest parties at the end of the execution.
    %
    Due to~\cref{lemma:apa}, there exist protocol parameterizations such that all honest parties share a common subset of at least size $m - \eta$, and since the output is picked as the median among a set with a majority of honest messages, all values in this subset stay in the convex hull of the input.
    %
    By following a similar argument as that in~\cref{lemma:one-third-apa}, it holds that for any (possibly the same) $\party_1, \ldots, \party_k \in \honestPartySet$,
    %
    \[ \min V \le \mathbf{V}^{\party_1}_{\eta + 1} \le \mathbf{V}^{\party_2}_{2\eta + 1} \le \ldots \le  \mathbf{V}^{\party_k}_{m - \eta} \le \max V. \]
    %
    Validity follows directly, and since all parties pick $1/\epsilon$ values after the $\mathsf{select}$ operation (\cref{eq:reduce-select}), the output quality is $\epsilon$.
\end{proof}
