\section{Permissionless Weak/Approximate Agreement}
\label{sec:reaching-weak-approx-agreement}

In this section we consider weaker forms of agreement --- namely, weak agreement and approximate agreement as introduced in~\cref{subsec:weak-approximate-agreement} --- in the permissionless setting, and show that, when appropriately parameterized, they can be achieved in constant time against a minority of corrupted computational power.
%
We present two protocols: first, a protocol that is simple and easy to understand, but which tolerates only one-third of corrupted computational power.
%
We then improve this protocol to tolerate a dishonest minority and, in the context of AA, achieve better output quality (recall~\cref{def:approximate-agreement}).\footnote{Translating protocols presented in this section to the PoS setting (with dynamic availability) is straightforward, as the core mechanism that we use---\mforone PoW---can be emulated by independently evaluating multiple different VRF outputs.}

For simplicity, we consider a single protocol invocation and present our protocols in the \emph{static} setting where parties are always online and their number is fixed yet unknown to the protocol participants, and we assume a global clock and a diffusion network with \delay-bounded delay.
%
Later, in~\cref{sec:permissionless-smr}, we use the weak/approximate agreement protocols as sub-routines in a dynamic environment with drifting clocks.

\subsection{Weak/Approximate Agreement against a $(1/3)$-Bounded Adversary}
\label{subsec:apa-one-third}

We first provide a simple protocol that solves Weak/Approximate Agreement against an adversary that controls up to one-third of the computational power.
%
At a high level, in this protocol parties first use \mforone PoW to mine and exchange messages containing their input value (concatenated with a random string for uniqueness) for a pre-determined number of $\ell$ rounds; then each party \party decides its output locally based on the messages that it has received.

The protocol is parameterized by $\ell$ (the termination time), $m$ (the number of parallel mining procedures) and $T$ (a pre-determined PoW target value).
%
A party \party starting with input $v$ queries the RO and gets $u = H(r \concat v)$ where $r$ is a $\kappa$-bit random string.
%
For every ``chunk'' \stringSegment{u}{i}{m} of $\kappa / m$ bits, \party checks if $\stringSegment{u}{i}{m} < T$.
%
If \party solves a PoW in any of the chunks, it diffuses $(r, v)$ to the network.
%
Note that $(r, v)$ will be counted independently for different succeeding chunks; e.g., if a valid PoW is found in $m / 2$ segments, $(r, v)$ will be counted for $m / 2$ times.
%
Parties keep listening to the network and book-keep all valid PoW messages, extracting their corresponding values in an array \textbf{V}.

Parties decide their output based on their local array \textbf{V} at the end of round $\ell$.
%
For Weak Agreement, they output $0$ ($1$ resp.) iff. more than two-thirds of the elements in \textbf{V} are $0$s ($1$s resp.); and $\bot$ otherwise.
%
Regarding Approximate Agreement, the output decision procedure then follows the approach in~\cite{JACM:DLPSW86}:
%
For an ordered array \textbf{V} of size $n$, the smallest and largest $n / 3$ values are dropped and the party outputs the midpoint of the remaining values (i.e., the average of the maximum and minimum element).

The full description of the protocol is presented in~\cref{protocol:one-third-approximate-consensus}.

\begin{cccProtocol}
    {$\mathsf{{\frac{1}{3}-}ApproxAgr}(\ell, m, T)$}
    {one-third-approximate-consensus}
    {Weak/Approxmate Agreeement tolerating a $(1 / 3)$-bounded adversary.}

    \begin{algorithmic}[1]
        \LineComment{The following code is executed in each round \round (\val is the party input)}
        \If{$\round \le \ell$} \Comment{Step 1: mine and exchange messages.}

        \State{Fetch information and denote incoming PoW messages by $((r_1, v_1), \ldots, (r_n, v_n))$}
        \For{each $(r_i, v_i)$}
        \State{Set $c = |\{j \mathbin| j \in [m] \wedge \stringSegment{H(r_i, v_i)}{j}{m} < T\}|$}
        \State{Append $v_i$ to $\mathbf{V}$ for $c$ times}
        \EndFor

        \State{Set $u \gets H(ctr, \val)$}
        \oneLineIf{$\exists j \in [m]$ s.t. $\stringSegment{u}{j}{m} < T$}{Diffuse $(ctr, \val)$}

        \State{$ctr \gets ctr + 1$}

        \Else  \Comment{Step 2: decide output.}

        \State{Order $\mathbf{V}$ non-decreasingly and denote $n \gets |\mathbf{V}|$} \label{code:ordered-array}

        \If{deciding output for weak agreement}
        \oneLineIf{$\mathbf{V}_{n / 3} = \mathbf{V}_{2n / 3}$}{output $\mathbf{V}_{n / 3}$ \textbf{else} output $\bot$}
        \ElsIf{deciding output for approximate agreement}
        \State{Output the mid point of $\mathsf{reduce}(\mathbf{V}, n / 3)$}
        \EndIf

        \EndIf
    \end{algorithmic}
\end{cccProtocol}

Note that different parties may work on different sets of messages when the adversary chooses to hold and deliver its own PoW message at the last round (so that honest parties have no time to diffuse this message to others).
%
Nonetheless, for sufficiently large $\kappa$ and suitable value of $\ell$, the messages generated by different parties are roughly proportional to their computational power.
%
Thus, for all arrays held by honest parties, they will share a large (yet unknown) common subset.

\begin{lemma} \label{lemma:one-third-apa}
    Let \delay denote the upper bound on network delay, $h$ and $t$ denote the number of honest and adversarial queries per round ($h > 2t$), and $m$ be the number of independent PoW mining.
    %
    Let $V$ denote the honest input set ($V = \{0, 1\}$ for Weak Agreement and $V \subseteq \mathbb{R}$ for Approximate Agreement) and $\mathbf{V}$ denote the ordered array of PoW messages received by party \party after round $\ell$ (line~\ref*{code:ordered-array} in~\cref{protocol:one-third-approximate-consensus}), and assume that $\ell > 4 \delay$ and $m = \bigTheta(\log^2 \kappa)$.
    %
    Then, for any two honest parties $\party, \party'$ (possibly $\party = \party'$) and $\mathbf{V}, \mathbf{V}'$ with size $n, n'$, respectively, it holds that
    %
    \[ \min V \le \mathbf{V}_{n / 3} \le \mathbf{V}'_{2n' / 3} \le \max V, \]
    %
    except with probability negligible in $\kappa$.
\end{lemma}

\begin{proof}
    Let $\mathbf{V}^h \subseteq \mathbf{V}$ denote the subset of messages generated by the honest parties before round $\ell - \delay$.
    %
    Let $\mathcal{H}$ denote the set of honest parties at the end of round $\ell$, and suppose it holds that
    %
    \begin{equation} \label{eq:one-third-aa}
        \forall \party_j \in \mathcal{H}, \Big| \bigcap_{\party_i \in \mathcal{H}} \mathbf{V}^h_i \Big| \ge 2 |\mathbf{V}_j| / 3,
    \end{equation}
    %
    then for $\mathbf{V}$ held by any honest party, it would hold that $\min V \le \mathbf{V}_{n / 3} \le \med(\bigcap_{\party_i \in \mathcal{H}} \mathbf{V}^h_i) \le \mathbf{V}_{2n / 3} \le \max V$, which would conclude the proof.

    Now we only need to verify \cref{eq:one-third-aa}.
    %
    Let $X$ denote the number of PoW messages honest parties generated in the first $(\ell - \delay)$ rounds, and $Y$ denote the number of successful PoWs the adversary generated in the first $\ell$ rounds, plus all the honest PoWs in the last \delay rounds.
    %
    Since $\ell > 4 \delay$, we have $\EX[X] > \EX[2Y]$.
    %
    Note that $\EX[X] = \bigTheta(\EX[2Y]) = \bigTheta(\log^2 \kappa)$.
    %
    Now, by letting $\epsilon \in (0, (\EX[X] / \EX[2Y] - 1) / 2 )$ be a constant and applying the Chernoff bound (\cref{thm:chernoff-bounds}), we get
    %
    \[ \Pr[X > 2Y] \ge \Pr[X > (1 - \epsilon) \EX[X]] \wedge \Pr[2Y < (1 + \epsilon) \EX[2Y]] \ge 1 - \exp(-\bigOmega(\log^2 \kappa)), \]
    %
    which happens with overwhelming probability in $\kappa$, the security parameter.
\end{proof}

Given the good properties satisfied by $\mathbf{V}$ as described in~\cref{lemma:one-third-apa}, we conclude that (i) by comparing $\mathbf{V}_{n / 3}$ and $\mathbf{V}_{2n / 3}$, honest parties reach Weak Agreement; and (ii) by applying the mid-point on \textsf{reduce}, honest parties reach Approximate Agreement with concentration quality $1/2$.
%
We prove this for Approximate Agreement in~\cref{thm:apprx-agr-one-third} (and Weak Agreement can be argued alike).

\begin{theorem} \label{thm:apprx-agr-one-third}
    Under the same assumption as in~\cref{lemma:one-third-apa}, \cref{protocol:one-third-approximate-consensus} is a $(1/2)$-secure Approximate Agreement protocol against an adversary that controls less than one-third of the computational power, except with negligible probability in the security parameter.
\end{theorem}

\begin{proof}
    Validity follows directly, since for any honest party $\party_i$, all values in $\mathsf{reduce}(\mathbf{V}_i, |\mathbf{V}_i| / 3)$ are within the convex hull of the honest inputs, and so is the midpoint.

    Regarding $(1/2)$-Agreement, note that there exist a value $u$ such that for any honest party \party and $\mathbf{V}$,  $\mathbf{V}_{n / 3} \le u \le \mathbf{V}_{2n / 3}$ (cf.~\cref{lemma:one-third-apa}).
    %
    For any two honest parties \party and $\party'$ and their respective outputs $v, v'$, let $v_1, v_2$ ($v'_1, v'_2$, resp.) denote the smallest and largest values in $\mathsf{reduce}(\mathbf{V}_i, |\mathbf{V}_i| / 3)$ ($\mathsf{reduce}(\mathbf{V}'_i, |\mathbf{V}'_i| / 3)$, resp.).
    %
    We have
    %
    \[ |v - v'| = (|v_1 - v'_1|  + |v_2 - v'_2|)/ 2 \le (\max V - u + u - \min V|) / 2 = (\max V - \min V) / 2, \]
    %
    which concludes the proof.
\end{proof}

\begin{remark}
    We note that the output quality of a single invocation of AA as \cref{protocol:one-third-approximate-consensus} can be improved when the adversary controls less computational power.
    %
    This is in line with the classical AA algorithm by Dolev \textit{et al.} \cite{JACM:DLPSW86}.
\end{remark}

The output quality $\epsilon$ can be improved by calling the same Approximate Agreement protocol sequentially, with each invocation using the output of the previous one as input, resulting in $n$ sequential calls improving the output quality to $\epsilon' = \epsilon^n$.
%
In the classical setting(s) (point-to-point channels, or PKI), since deterministic termination is guaranteed, the sequential composition of AA protocols is trivial.
%
In a permissionless environment, however, where only a public setup is available, the sequential composition becomes more challenging in that a common reference string is needed for \emph{every} invocation in order to avoid pre-mining.
%
This problem has been addressed in~\cite{EC:GarKiaShe24} for the case of synchronous protocols (i.e., no [unknown] bounded delay) with static number of participants.
%
In~\cref{sec:permissionless-smr}, we show how this mechanism can be adapted to the drifting clock model with fluctuating number of parties.

\subsection{Weak/Approximate Agreement with Optimal Corruption Threshold}
\label{subsec:apa-honst-majority}

\cref{protocol:one-third-approximate-consensus} works only when the adversarial computational power is bounded by one-third, and in terms of AA the protocol offers concentration quality $\epsilon = 1 / 2$.
%
In this section we present a new protocol that improves the corruption threshold to honest majority for both Weak and Approximate Agreement.
%
Moreover, for Approximate Agreement, it achieves $\epsilon$-Agreement for an arbitrarily small $\epsilon > 0$ after a single invocation.

We first provide some intuition on why~\cref{protocol:one-third-approximate-consensus} fails in an honest majority setting, taking Approximate Agreement as an example.
%
Notice that with stronger adversarial computational power, the intersection of PoW message arrays held by an honest party at the decision phase becomes too small to account for two-thirds of its PoW message array.
%
Thus, since the common subset is not large enough, the \textsf{reduce} and mid-point computations can no longer guarantee $\epsilon$-Agreement for $\epsilon = 1 / 2$ and Validity at the same time.
%
In more detail, when the adversarial messages (and inconsistent honest messages due to network delay) can account for more than one-third of the messages, parties need to trim more values from both sides of the ordered array; otherwise, Validity would not hold because the mid-point computation may take some adversarially proposed values as input.
%
However, trimming more than one-third of the elements from both sides hurts $\epsilon$-Agreement, as the inequalities in~\cref{lemma:one-third-apa} no longer stand.

To circumvent the above situation, our solution is to categorize the different PoW messages based on their indices (in the $m$ parallel procedures) where they succeed, and select one message in each procedure to form a message array \textbf{V} of fixed size of $m$ values, thus providing a more refined approach to decide on an output value.

\paragraph{Protocol description.}
%
Here we show how to extend \cref{protocol:one-third-approximate-consensus} to the honest majority setting.
%
The protocol takes two additional parameters $k$ and $\eta$, which are explained below, and runs $m$ PoW-based Byzantine agreement (BA) procedures in parallel (cf.~\cite{EC:GarKiaLeo15})\footnote{We note that, while running a single invocation of such BA procedure \cite{EC:GarKiaLeo15} for $\bigTheta(\log^2 \kappa)$ rounds yields a full agreement, terminating after constantly many rounds does not give any form of weak agreement with overwhelming probability.}.
%
In each procedure, it builds a PoW-based blockchain (which at a high level follows the Nakamoto protocol) and binds the mining procedure of the chain with ``input-blocks'' using \twoforone PoWs (thus, each party maintains $m$ parallel blockchains which we denote by \parallelChains).
%
Specifically, for a RO query output $u$ and each chain index $i \in [m]$, alongside with checking if a block that is used to extend the chain is produced by evaluating $\stringSegment{u}{i}{m} < T$, the \twoforone mining further evaluates the reverse of this chunk \stringSegmentRev{u}{i}{m}.
%
If $\stringSegmentRev{u}{i}{m} < T$, an input-block \inputBlock containing the miner's input is mined and \inputBlock can be included in the corresponding $i$-th blockchain.
%
Yet, a single RO query can produce different valid blocks to extend different blockchains, as well as an input block that can be valid on multiple chains.
%
Refer to~\cref{protocol:approximate-consensus} Line~\ref*{code:mforone-pow-mining-begin} to \ref*{code:mforone-pow-mining-end} to see how the \mforone and \twoforone mining procedures are bound together in order to get $m$ independent parallel instances of the PoW-based BA protocol.
%
Also note that, parties extend each blockchain \chain independent following the longest chain selection rule (as specified in the Bitcoin backbone protocol~\cite{EC:GarKiaLeo15} which we omit the details on chain validation and selection here), and keeps including valid and unique input-blocks with respect to \chain.

After $\ell$ rounds, for each parallel chain $\chain_i$, parties extract all input-blocks in its prefix by pruning blocks based on their timestamp, subject to the ``common prefix'' parameter $k$.\footnote{In ordinary blockchains, after pruning $k$ blocks, parties hold a chain that is a prefix of any other party's with overwhelming probability, hence the term ``common prefix.'' However, since our protocol terminates in constant time, this pruning operation will lead to a common prefix only with constant probability.}
%
Then, these input-blocks are ordered based on their value contained and the median one is picked as the output of chain $\chain_i$.
%
This forms an array $\mathbf{V}$ of size $m$.
%
For Approximate Agreement, each party then decides its output based on its local $\mathbf{V}$ and a parameter $\eta < m / 2$, by first ordering $\mathbf{V}$ and removing the smallest and largest $\eta$ values, selecting the first element of every $\eta$ elements, and computing the average of the selected ones.
%
I.e., they output $\textsf{avg}(\mathsf{select}(\mathsf{reduce}(\mathsf{V}, \eta), \eta))$ and terminate.
%
In terms of Weak Agreement, parties output a value that accounts for a super-majority of the chains if it exists, and $\mathcal{\bot}$ otherwise.

\begin{cccProtocol}
    {$\mathsf{{\frac{1}{2}-}ApproxAgr}(\ell, m, T, k, \eta)$}
    {approximate-consensus}
    {Honest-majority Weak/Approxmate Agreeement.}

    \begin{algorithmic}[1]
        \LineComment{The following code is executed in each round \round (\val is the party input)}

        \If{$r \le \ell$}
        \State{Fetch incoming chains $(\chain_1, \ldots, \chain_N)$}

        \State{$\parallelChainsLocal \gets \mathsf{UpdateLocalChain}(\parallelChainsLocal, \chain_1, \ldots, \chain_N)$}
        \Comment{Select longest valid chains for each index}

        \State{Fetch incoming input-blocks $(\inputBlock_1, \ldots, \inputBlock_k)$}
        \State{Add valid $\inputBlock_1, \ldots, \inputBlock_{k'}$ to \buffer}

        \LineComment{\mforone PoW mining}

        \State{$h \gets \varepsilon$, $st \gets \varepsilon$} \label{code:mforone-pow-mining-begin}
        \For{$i = 1$ \textbf{to} $m$}
        \State{$h \gets h \concat \stringSegment{H(\chainHead{\parallelChains_i})}{i}{m}$}
        \State{Let $\buffer_i$ denote all \inputBlock that are valid yet not included w.r.t. $\parallelChains_i$}

        \State{$st \gets st \concat \blockify(\buffer_i)$\footnote{\blockify translates a sequence of transactions to the ledger state (cf.~\cite{C:BMTZ17}).}}

        \EndFor
        \State{$u \gets H(ctr, r, h, st, val)$}

        \For{$i = 1$ \textbf{to} $m$}
        \Comment{Check if PoW succeeds on any type of block.}
        \oneLineIf{$\stringSegment{u}{i}{m} < T$}{set $\parallelChains_i \gets \parallelChains_i \concat  \langle ctr, r, h, st, val \rangle$ and diffuse $\parallelChains_i$} \Comment{Extend chain}
        \oneLineIf{$\stringSegmentRev{u}{i}{m} < T$}{diffuse  $\inputBlock = \langle ctr, r, h, st, val \rangle$}
        \EndFor
        \State{$ctr \gets ctr + 1$} \label{code:mforone-pow-mining-end}

        \Else

        \State{Initialize $\mathbf{V}$ to an empty array}

        \For{$i$ \textbf{from} $1$ \textbf{to} $m$}
        \Comment{Extract output from parallel chains}

        \State{Initialize $\mathbf{M}$ to an empty array}
        \For{$\inputBlock \in \block \in \chainPrefixUB{\parallelChains_i}{\ell - k}$}
        \State{Parse \inputBlock as $\langle \cdot, \cdot, \cdot, \cdot, val \rangle$ and add $val$ to $\mathbf{M}$}
        \EndFor
        \State{Sort $\mathbf{M}$ and add $\med (\mathbf{M})$ to $\mathbf{V}$}
        \EndFor

        \LineComment{Decide final output}
        \If{deciding output for weak agreement}
        \State{Order $\mathbf{V}$ non-decreasingly and denote $n \gets |\mathbf{V}|$}
        \oneLineIf{$\mathbf{V}_\eta = \mathbf{V}_{n - \eta}$}{output $\mathbf{V}_{\eta}$ \textbf{else} output $\bot$}
        \ElsIf{deciding output for approximate agreement}
        \State{Output $\textsf{avg}(\mathsf{select}(\mathsf{reduce}(\mathbf{V}, \eta), \eta))$}
        \EndIf
        \EndIf
    \end{algorithmic}
\end{cccProtocol}


Next, in the following lemma we show that running the $m$ parallel chains and terminating in a constant number of rounds in a bounded delay network yields good properties on a fraction of the chains.
%
The two properties in~\cref{lemma:apa} have been proven for a synchronous network in~\cite{EC:GarKiaShe24}.
%
Here we extend this result to the bounded-delay network setting.

\begin{lemma} \label{lemma:apa}
    There exist parameterizations of~\cref{protocol:approximate-consensus} such that the following holds.
    %
    Let \parallelChains and $\parallelChains'$ denote the parallel chains held by two honest parties \party and $\party'$ at the end of round $\ell$, respectively.
    %
    There exists a subset $S \subseteq \{1, 2, \ldots, m \}$, $|S| > m - \eta$, such that for all $i \in S$, the following properties hold on chains $\chain = \parallelChains_i$ and $\chain' = \parallelChains'_i$:
    %
    \begin{cccItemize}[nosep]
        \item \emph{\bf Agreement}: $\chainPrefixUB{\chain}{\ell - k} = \chainPrefixUB{\chain'}{\ell - k}$.

        \item \emph{\bf Honest input-block majority}: More than half of the input-blocks included in $\chainPrefixUB{\chain}{\ell - k}$ and $\chainPrefixUB{\chain'}{\ell - k}$ are produced by honest parties.
    \end{cccItemize}
\end{lemma}

\begin{proof}[Proof (sketch)]
    Consider the $i$-th single chain $\chain = \parallelChains_i$ of a party \party.
    %
    There exist parameterizations such that at the end of round $\ell$, with constant probability $p > (m - \eta) / m$, $\chainPrefixUB{\chain}{\ell - k}$ yields a common view with any other honest party and includes a majority of honest input blocks.
    %
    Proving the above claim is mainly a reminiscence of the proof of~\cite[Theorem 2]{EC:GarKiaShe24}, with additionally extending it to the bounded-delay network (see, e.g., \cite[Section 7]{EPRINT:GarKiaLeo14}, for more details).

    Since any two single chains are mutually independent from each other, the probability that these two % good 
    properties hold on more than $p \cdot m  > m - \eta$ chains can be computed by the Chernoff bound (\cref{thm:chernoff-bounds}), with error probability negligibly small in the security parameter $\kappa$ (recall that $m = \polylog \kappa$).
\end{proof}

Given that the `good' properties hold on a sufficiently large subset of the values in the output decision array, we conclude that \cref{protocol:approximate-consensus} solves the Approximate Agreement problem with an arbitrarily small $\epsilon$.

\begin{theorem} \label{thm:honest-majority-approx-agreement}
    There exist protocol parameterizations such that \cref{protocol:approximate-consensus} is an $\epsilon$-secure Approximate Agreement protocol, for an arbitrarily small $\epsilon > 0$, against an adversary that controls less than half of the computational power, and all honest parties terminate at the end of round $\ell$, except with probability negligible in $\kappa$.
\end{theorem}

\begin{proof}
    Let $\eta = [\epsilon/(1 + 2\epsilon)]m$ and consider the arrays of size $m$ held by honest parties at the end of the execution.
    %
    Due to~\cref{lemma:apa}, there exist protocol parameterizations such that all honest parties share a common subset of at least size $m - \eta$, and since the output is picked as the median among a set with a majority of honest messages, all values in this subset stay in the convex hull of the input.
    %
    By following a similar argument as that in~\cref{lemma:one-third-apa}, it holds that for any (possibly the same) $\party_1, \ldots, \party_k \in \honestPartySet$,
    %
    \[ \min V \le \mathbf{V}^{\party_1}_{\eta + 1} \le \mathbf{V}^{\party_2}_{2\eta + 1} \le \ldots \le  \mathbf{V}^{\party_k}_{m - \eta} \le \max V. \]
    %
    Validity follows directly, and since all parties pick $1/\epsilon$ values after the $\mathsf{select}$ operation (\cref{eq:reduce-select}), the output quality is $\epsilon$.
\end{proof}

